\documentclass{article}

\usepackage{calc}
\usepackage{calculator}
\usepackage{geometry}
\usepackage[figuresleft]{rotating} % 'figuresleft' rotates the content of a 'sidewaysfigure'-page to the right
\usepackage{fancyhdr}
\usepackage{datetime2}
\usepackage[hidelinks]{hyperref} % 'hidelinks' prevents underscores & borders around links
\usepackage[dvipsnames]{xcolor}
\usepackage{tcolorbox}
\usepackage{tikz}
\usepackage{etoolbox}
\usepackage{siunitx}
\usepackage{enumitem}
\usepackage{amsmath}
\usepackage{pdflscape}

% PAGE GEOMETRY
\geometry{a4paper,
          landscape,
          top=1cm,
          bottom=1cm,
          left=1cm,
          right=1cm,
          includehead,
          includefoot}
\pagestyle{fancy}
\setlength{\headheight}{18pt}
\tcbuselibrary{breakable, raster, skins}
\usetikzlibrary{calc, matrix, backgrounds}
% END PAGE GEOMETRY

% META
\newcommand{\thetitle}{Anno 1800 Prod Chains \& Consumption}
\newcommand{\theauthor}{Michael Stocker (@dot\_Sp0T)}
\newcommand{\theauthorsgithub}{https://github.com/dotSp0T}
\newcommand{\theversion}{v0.2}
\newcommand{\thelicense}{CC BY-SA 4.0}
\newcommand{\thelicenseurl}{https://creativecommons.org/licenses/by-sa/4.0/}
\newcommand{\thesourcerepository}{https://github.com/dotSp0T/Anno_1800_Chains_Consumption}

\newcommand{\theinfo}{! Pop- \& Prod-Numbers are for single factories !}
\newcommand{\thewiki}{https://anno1800.fandom.com}

% texts
\newcommand{\oldworld}{\smallicon{Old_World}}
\newcommand{\newworld}{\smallicon{New_World}}
\newcommand{\arctic}{\smallicon{Arctic}}
\newcommand{\enbesa}{\smallicon{Enbesa}}
\newcommand{\capetrelawney}{\smallicon{Cape_Trelawney}}

% sizes
\newcommand{\unitsize}{1cm}
\newcommand{\boxheight}{3.25cm}

% raster
\newcommand{\landscaperaster}{9}
\newcommand{\portraitraster}{6}

% position offsets
\newcommand{\facnumdist}{-(0.3*\unitsize)}
\newcommand{\facnumdegs}{95}
\newcommand{\lblmrkdist}{-(0.35*\unitsize)}
\newcommand{\lblmrkdegs}{65}
\newcommand{\facproddist}{-(0.2*\unitsize)}
\newcommand{\facproddegs}{0}
\newcommand{\fldnumdist}{-(0.2*\unitsize)}
\newcommand{\fldnumdegs}{270}
\newcommand{\stratalistmath}{-(0.55*\unitsize*\arabic{strata}) + (0.25*\unitsize*(\arabic{strata} - 2))}

% nodes
\newcommand{\nodepadding}{-2pt}

% tikz
\tikzset{
    hyperlink node/.style={% hyperlink style, https://tex.stackexchange.com/a/36111/186115
        alias=sourcenode,
        append after command={
            let \p1 = (sourcenode.north west),
                \p2=(sourcenode.south east),
                \n1={\x2-\x1},
                \n2={\y1-\y2} in
            node [inner sep=0pt, outer sep=0pt,anchor=north west,at=(\p1)] {\href{#1}{\XeTeXLinkBox{\phantom{\rule{\n1}{\n2}}}}}
                    %xelatex needs \XeTeXLinkBox, won't create a link unless it
                    %finds text --- rules don't work without \XeTeXLinkBox.
                    %Still builds correctly with pdflatex and lualatex
        }
    }% END hyperlink style
}

% P R O D N O D E
\newcommand{\prodpermin}[1]{ % #1 decimal number to be rounded to 3 digits
    \DIVIDE{60}{#1}{\sol}
    \num[round-mode=places,round-precision=3]{\sol}%
}

% #1 (x,y)
% #2 Object_Name (filename without .png)
% #3 Amount
% #4 anno1800 wiki path (../wiki/<HERE>)
\newcommand{\prodnode}[4]{
    \node[outer ysep=\nodepadding,
          label={[label distance=\facnumdist]\facnumdegs:#3×},
          hyperlink node=\thewiki/wiki/#4] (#2) at #1 {\icon{#2}};
}

% #5 node identifier
% #6 label (e.g. New World)
\newcommand{\prodnodelbld}[6]{
    \node[outer ysep=\nodepadding,
          label={[label distance=\facnumdist]\facnumdegs:#3×},
          label={[label distance=\lblmrkdist]\lblmrkdegs:#6},
          hyperlink node=\thewiki/wiki/#4] (#5) at #1 {\icon{#2}};
}


% #5 amount of units produced per factory per minute
\newcommand{\prodnodec}[5]{
    \node[outer ysep=\nodepadding,
          label={[label distance=\facnumdist]\facnumdegs:#3×},
          label={[label distance=\facproddist]\facproddegs:@\prodpermin{#5}/min},
          hyperlink node=\thewiki/wiki/#4] (#2) at #1 {\icon{#2}};
}

% #3 Amount of electrified factories
% #5 node identifier
% #6 label (e.g. New World)
\newcommand{\prodnodelbldfe}[6]{
    \node[outer ysep=\nodepadding,
          label={[label distance=\facnumdist]\facnumdegs:\smallicon{Electricity}\textcolor{SkyBlue}{#3×}},
          label={[label distance=\lblmrkdist]\lblmrkdegs:#6},
          hyperlink node=\thewiki/wiki/#4] (#5) at #1 {\icon{#2}};
}

% #3 Amount of electrified factories
\newcommand{\prodnodefe}[4]{
	\node[outer ysep=\nodepadding,
	label={[label distance=\facnumdist]\facnumdegs:\smallicon{Electricity}\textcolor{SkyBlue}{#3×}},
	hyperlink node=\thewiki/wiki/#4] (#2) at #1 {\icon{#2}};
}

% #3 Amount of electrified factories
% #5 amount of units produced per factory per minute
\newcommand{\prodnodefec}[5]{
    \node[outer ysep=\nodepadding,
    label={[label distance=\facnumdist]\facnumdegs:\smallicon{Electricity}\textcolor{SkyBlue}{#3×}},
          label={[label distance=\facproddist]\facproddegs:\textcolor{SkyBlue}{@\prodpermin{#5}/min}},
          hyperlink node=\thewiki/wiki/#4] (#2) at #1 {\icon{#2}};
}

% F A R M N O D E
% #1 (x,y)
% #2 Object_Name (filename without .png)
% #3 Amount
% #4 anno1800 wiki path (../wiki/<HERE>)
% #5 field-size (e.g. 1×1)
% #6 amount of fields needed (e.g. 72)
\newcommand{\farmnode}[6]{
    \node[outer ysep=\nodepadding,
          label={[label distance=\facnumdist]\facnumdegs:#3×},
          label={[label distance=\fldnumdist]\fldnumdegs:#6 (#5)},
          hyperlink node=\thewiki/wiki/#4] (#2) at #1 {\icon{#2}};
}

% #7 node identifier
% #8 label (e.g. New World)
\newcommand{\farmnodelbld}[8]{
    \node[outer ysep=\nodepadding,
          label={[label distance=\facnumdist]\facnumdegs:#3×},
          label={[label distance=\fldnumdist]\fldnumdegs:#6 (#5)},
          label={[label distance=\lblmrkdist]\lblmrkdegs:#8},
          hyperlink node=\thewiki/wiki/#4] (#7) at #1 {\icon{#2}};
}

% #7 source farm
\newcommand{\farmnodep}[7]{
	\node[outer ysep=\nodepadding,
	label={[label distance=\facnumdist]\facnumdegs:\smallicon{Electricity}\textcolor{SkyBlue}{#3×}},
	label={[label distance=\fldnumdist]\fldnumdegs:#6 (#5)},
	label={[label distance=\lblmrkdist]\lblmrkdegs:\smallicon{#7}},
	hyperlink node=\thewiki/wiki/#4] (#2) at #1 {\icon{#2}};
}

% #1 (x,y)
% #2 Object_Name (filename without .png)
% #3 Amount
% #4 anno1800 wiki path (../wiki/<HERE>)
% #5 field-size (e.g. 1×1)
% #6 amount of fields needed (e.g. 72)
% #7 Comma Separated list: <stratum>:<amount>,<stratum>:<amount>, ...
\newcommand{\farmnodec}[7]{
	\consstratalist{#7}
	\node[outer sep=-1mm,
	label={[label distance=\facnumdist]\facnumdegs:#3×},
	label={[label distance=\fldnumdist]\fldnumdegs:#6 (#5)},
	hyperlink node=\thewiki/wiki/#4] (#2) at #1 {
	\begin{tikzpicture}
		\node[outer ysep=\nodepadding] at (0,0) {\icon{#2}};
		\matrix[inner sep=2pt,
		row sep=-2pt,
		column sep=-1mm,
		matrix of nodes,
		ampersand replacement=\&] (output) at (0,{\stratalistmath}) {
			\matrixcontents};
	\end{tikzpicture}};
}




% MultiFac Node
% #1 (x,y)
% #2 Object_Name (filename without .png)
% #3 Amount
% #4 Multi_Type
% #5 Multi_Type(link)
\newcommand{\multifac}[5]{
	\node[outer ysep=\nodepadding,
	label={[label distance=\facnumdist]\facnumdegs:#3×},
	label={[label distance=\lblmrkdist]\lblmrkdegs:\smallicon{#4}},
	hyperlink node=\thewiki/wiki/#5] (#2) at #1 {\icon{#2}};
}
% MultiFac Node
% #6 Comma Separated list: <stratum>:<amount>,<stratum>:<amount>, ...
\newcommand{\multifaccons}[6]{
	\consstratalist{#6}
	\node[outer ysep=\nodepadding,
	label={[label distance=\facnumdist]\facnumdegs:#3×},
	label={[label distance=\lblmrkdist]\lblmrkdegs:\smallicon{#4}},
	hyperlink node=\thewiki/wiki/#5] (#2) at #1 {
		\begin{tikzpicture}
			\node[outer ysep=\nodepadding] at (0,0) {\icon{#2}};
			\matrix[inner sep=2pt,
			row sep=-2pt,
			column sep=-1mm,
			matrix of nodes,
			ampersand replacement=\&] (#2) (output) at (0,{\stratalistmath}) {
				\matrixcontents};
	\end{tikzpicture}};;
}
% MultiFac Node Production
% #6 Prod per min
\newcommand{\multifacc}[6]{
	\node[outer ysep=\nodepadding,
	label={[label distance=\facnumdist]\facnumdegs:#3×},
	label={[label distance=\facproddist]\facproddegs:@\prodpermin{#6}/min},
	label={[label distance=\lblmrkdist]\lblmrkdegs:\smallicon{#4}},
	hyperlink node=\thewiki/wiki/#5] (#2) at #1 {\icon{#2}};
}

% Old World ORCHARD NODE
% #1 (x,y)
% #2 Object_Name (filename without .png)
% #3 Amount
\newcommand{\orchnodeold}[3]{ \multifac{#1}{#2}{#3}{Orchard_Old}{Orchard} }
% Old_World ORCHARD NODE End
% #4 Comma Separated list: <stratum>:<amount>,<stratum>:<amount>, ...
\newcommand{\orchnodeoldcons}[4]{ \multifaccons{#1}{#2}{#3}{Orchard_Old}{Orchard}{#4} }
% New World ORCHARD NODE
% #1 (x,y)
% #2 Object_Name (filename without .png)
% #3 Amount
\newcommand{\orchnodenew}[3]{ \multifac{#1}{#2}{#3}{Orchard_New}{Orchard} }

% Old World Chemical Plant NODE
% #1 (x,y)
% #2 Object_Name (filename without .png)
% #3 Amount
\newcommand{\chemnodeold}[3]{ \multifac{#1}{#2}{#3}{Chemical_Plant_Old}{Chemical_Plant} }
% Old_World Chemical Plant NODE End
% #4 Comma Separated list: <stratum>:<amount>,<stratum>:<amount>, ...
\newcommand{\chemnodeoldcons}[4]{ \multifaccons{#1}{#2}{#3}{Chemical_Plant_Old}{Chemical_Plant}{#4} }
\newcommand{\chemnodeoldc}[4]{ \multifacc{#1}{#2}{#3}{Chemical_Plant_Old}{Chemical_Plant}{#4} }
% New World Chemical Plant NODE
% #1 (x,y)
% #2 Object_Name (filename without .png)
% #3 Amount
\newcommand{\chemnodenew}[3]{ \multifac{#1}{#2}{#3}{Chemical_Plant_New}{Chemical_Plant} }
% New_World Chemical Plant NODE End
% #4 Comma Separated list: <stratum>:<amount>,<stratum>:<amount>, ...
\newcommand{\chemnodenewcons}[4]{ \multifaccons{#1}{#2}{#3}{Chemical_Plant_New}{Chemical_Plant}{#4} }
% New_World Chemical Plant NODE End
% #4 Comma Separated list: <stratum>:<amount>,<stratum>:<amount>, ...
\newcommand{\chemnodenewc}[4]{ \multifacc{#1}{#2}{#3}{Chemical_Plant_New}{Chemical_Plant}{#4} }

% AssemblyLine NODE
% #1 (x,y)
% #2 Object_Name (filename without .png)
% #3 Amount
\newcommand{\asslnode}[3]{ \multifac{#1}{#2}{\smallicon{Electricity}\textcolor{SkyBlue}{#3}}{AssemblyLine}{AssemblyLine} }
% AssemblyLine NODE End
% #4 Comma Separated list: <stratum>:<amount>,<stratum>:<amount>, ...
\newcommand{\asslnodecons}[4]{ \multifaccons{#1}{#2}{\smallicon{Electricity}\textcolor{SkyBlue}{#3}}{AssemblyLine}{AssemblyLine}{#4} }
\newcommand{\asslnodec}[4]{ \multifacc{#1}{#2}{\smallicon{Electricity}\textcolor{SkyBlue}{#3}}{AssemblyLine}{AssemblyLine}{#4} }
% Artisans_Workshopt NODE
% #1 (x,y)
% #2 Object_Name (filename without .png)
% #3 Amount
\newcommand{\arwsnode}[3]{ \multifac{#1}{#2}{#3}{Artisans_Workshop}{Artisans_Workshop} }
% Artisans_Workshop NODE End
% #4 Comma Separated list: <stratum>:<amount>,<stratum>:<amount>, ...
\newcommand{\arwsnodecons}[4]{ \multifaccons{#1}{#2}{#3}{Artisans_Workshop}{Artisans_Workshop}{#4} }

% Recipe Node
% #1 (x,y)
% #2 Object_Name (filename without .png)
% #3 Amount 
% #4 Venue link
% #5 Comma Separated list: <stratum>:<amount>,<stratum>:<amount>, ...
\newcommand{\venue}[5]{
	\conseffectlist{#5}
	\node[outer ysep=\nodepadding,
	label={[label distance=\lblmrkdist]\facnumdegs:#3x},
	label={[label distance=-(0.55*\unitsize)]80:\smallicon{#4}},
	hyperlink node=\thewiki/wiki/#4] (#2) at #1 {
		\begin{tikzpicture}
			\node[outer ysep=\nodepadding] at (0,0) {\icon{#2}}; 
			\matrix[inner sep=2pt,
			row sep=-2pt,
			column sep=-1mm,
			matrix of nodes,
			ampersand replacement=\&] (#2) (output) at (0,{\stratalistmath}) {
				\matrixcontents};
	\end{tikzpicture}};;
}
\newcommand{\restrecipe}[4]{ \venue{#1}{#2}{#3}{Restaurant}{#4} }
\newcommand{\barrecipe}[4]{  \venue{#1}{#2}{#3}{Bar}{#4} }
\newcommand{\caferecipe}[4]{ \venue{#1}{#2}{#3}{Cafe}{#4} }
\newcommand{\ironrecipe}[4]{ \venue{#1}{#2}{#3}{The_Iron_Tower}{#4} }
\newcommand{\drugpatent}[4]{ \venue{#1}{#2}{#3}{Drug_Store}{#4} }
\newcommand{\deppatent}[4]{ \venue{#1}{#2}{#3}{Department_Store}{#4} }
\newcommand{\furpatent}[4]{ \venue{#1}{#2}{#3}{Furniture_Store}{#4} }


% C O N S N O D E
\newcounter{strata}
\newcommand{\conseffectlist}[1]{
    \let\matrixcontents\empty
    \setcounter{strata}{0}
    \foreach \l/\r in {#1}{
        \expandafter\gappto\expandafter\matrixcontents\expandafter{\expandafter\smallicon\expandafter{\l} \&}
        \expandafter\gappto\expandafter\matrixcontents\expandafter{\r \\}
        \stepcounter{strata}
    }
}

\newcommand{\consstratalist}[1]{
	\let\matrixcontents\empty
	\setcounter{strata}{0}
	\foreach \p\s\w\h\i in {#1}{
		\expandafter\gappto\expandafter\matrixcontents\expandafter{\expandafter\smallicon\expandafter{\p} \&}
		\expandafter\gappto\expandafter\matrixcontents\expandafter{\s \&}
		\expandafter\ifstrequal\expandafter{\w}{0}{}{
			\expandafter\gappto\expandafter\matrixcontents\expandafter{\expandafter\smallicon\expandafter{\p _Workforce} \&}
			\expandafter\gappto\expandafter\matrixcontents\expandafter{\w \&}
		}
		\expandafter\ifstrequal\expandafter{\h}{0}{}
		{
	       \expandafter\ifstrequal\expandafter{\p}{Explorers}
			{\expandafter\gappto\expandafter\matrixcontents\expandafter{\expandafter\smallicon\expandafter{Heater} \&}}
			{\expandafter\ifstrequal\expandafter{\p}{Technicians}
				{\expandafter\gappto\expandafter\matrixcontents\expandafter{\expandafter\smallicon\expandafter{Heater} \&}}	
				{\expandafter\gappto\expandafter\matrixcontents\expandafter{\expandafter\smallicon\expandafter{Happiness_positive} \&}}
			}
			\expandafter\gappto\expandafter\matrixcontents\expandafter{\h \&}
		}
		\expandafter\ifstrequal\expandafter{\i}{0}
		{\expandafter\gappto\expandafter\matrixcontents{\\}}
		{
			\expandafter\ifstrequal\expandafter{\p}{Scholars}
			{\expandafter\gappto\expandafter\matrixcontents\expandafter{\expandafter\smallicon\expandafter{Research} \&}}
			{\expandafter\gappto\expandafter\matrixcontents\expandafter{\expandafter\smallicon\expandafter{Credits} \&}}
			\expandafter\gappto\expandafter\matrixcontents\expandafter{\i \\}
		}
		\stepcounter{strata}
	}
}

% #1 (x,y)
% #2 Object_Name (filename without .png)
% #3 Amount
% #4 anno1800 wiki path (../wiki/<HERE>)
% #5 Comma Separated list: <stratum>:<amount>,<stratum>:<amount>, ...
\newcommand{\consnode}[5]{
    \consstratalist{#5}

    \node[outer sep=-1mm,
          label={[label distance=\facnumdist]\facnumdegs:#3×},
          hyperlink node=\thewiki/wiki/#4] (#2) at #1 {
        \begin{tikzpicture}
            \node[outer ysep=\nodepadding] at (0,0) {\icon{#2}};
            \matrix[inner sep=2pt,
                    row sep=-2pt,
                    column sep=-1mm,
                    matrix of nodes,
                    ampersand replacement=\&] (output) at (0,{\stratalistmath}) {
                \matrixcontents};
        \end{tikzpicture}};
}

% #5 amount of units produced per factory per minute
% #6 Comma Separated list: <stratum>:<amount>,<stratum>:<amount>, ...
\newcommand{\consnodec}[6]{
	\consstratalist{#6}
	
	\node[outer sep=-1mm,
	label={[label distance=\facnumdist]\facnumdegs:#3×},
	label={[label distance=\facproddist]\facproddegs:@\prodpermin{#5}/min},
	hyperlink node=\thewiki/wiki/#4] (#2) at #1 {
		\begin{tikzpicture}
			\node[outer ysep=\nodepadding] at (0,0) {\icon{#2}};
			\matrix[inner sep=2pt,
			row sep=-2pt,
			column sep=-1mm,
			matrix of nodes,
			ampersand replacement=\&] (output) at (0,{\stratalistmath}) {
				\matrixcontents};
	\end{tikzpicture}};
}

% #1 (x,y)
% #2 Object_Name (filename without .png)
% #3 Amount attrativeness
\newcommand{\cultnode}[3]{
	\node[outer ysep=\nodepadding,
	label={[label distance=\fldnumdist]\fldnumdegs:\textcolor{olive}{#3}}]
	 (#2) at #1 {\icon{#2}};
}

% #1 (x,y)
% #2 Object_Name (filename without .png)
% #3 Amount attrativeness
% #4 musicsoreyear
\newcommand{\cultnodelbld}[4]{
	\node[outer ysep=\nodepadding,
	label={[label distance=\lblmrkdist]\lblmrkdegs:\smallicon{music_score_#4}},
	label={[label distance=\fldnumdist]\fldnumdegs:\textcolor{olive}{#3}}]
	(#2) at #1 {\icon{#2}};
}

% #1 (x,y)
% #2 Object_Name (filename without .png)
% #3 Amount
% #4 anno1800 wiki path (../wiki/<HERE>)
% #5 node identifier
% #6 label (e.g. New World)
% #7 Comma Separated list: <stratum>:<amount>,<stratum>:<amount>, ...
\newcommand{\consnodelbld}[7]{
	\consstratalist{#7}
	
	\node[outer sep=-1mm,
	label={[label distance=\facnumdist]\facnumdegs:#3×},
	label={[label distance=\lblmrkdist]\lblmrkdegs:#6},
	hyperlink node=\thewiki/wiki/#4] (#5) at #1 {
		\begin{tikzpicture}
			\node[outer ysep=\nodepadding] at (0,0) {\icon{#2}};
			\matrix[inner sep=2pt,
			row sep=-2pt,
			column sep=-1mm,
			matrix of nodes,
			ampersand replacement=\&] (output) at (0,{\stratalistmath}) {
				\matrixcontents};
	\end{tikzpicture}};
}

% #3 Amount of electrified factories
\newcommand{\consnodefe}[5]{
    \consstratalist{#5}

    \node[outer sep=-1mm,
          label={[label distance=\facnumdist]\facnumdegs:\smallicon{Electricity}\textcolor{SkyBlue}{#3×}},
          hyperlink node=\thewiki/wiki/#4] (#2) at #1 {
        \begin{tikzpicture}
            \node[outer ysep=\nodepadding] at (0,0) {\icon{#2}};
            \matrix[inner sep=2pt,
                    row sep=-2pt,
                    column sep=-1mm,
                    matrix of nodes,
                    ampersand replacement=\&] (output) at (0,{\stratalistmath}) {
                \matrixcontents};
        \end{tikzpicture}};
}

% #1 (x,y)
% #2 Object_Name (filename without .png)
% #3 Amount
% #4 field-size (e.g. 1×1)
% #5 amount of fields needed (e.g. 72)
\newcommand{\hacfarmnode}[5]{\farmnodelbld{#1}{#2}{#3}{Hacienda_Farm}{#4}{#5}{#2}{\smallicon{Hacienda}}}

% #1 (x,y)
% #2 Object_Name (filename without .png)
% #3 Amount
\newcommand{\hacbrewnode}[3]{\prodnodelbld{#1}{#2}{#3}{Hacienda_Brewery}{#2}{\smallicon{Hacienda}}}

% Recipe Node
% #1 (x,y)
% #2 Object_Name (filename without .png)
% #3 Amount 
% #4 Comma Separated list: <stratum>:<amount>,<stratum>:<amount>, ...
\newcommand{\hacbrewnodecons}[4]{\consnodelbld{#1}{#2}{#3}{Hacienda_Brewery}{#2}{\smallicon{Hacienda}}{#4}}

\newcommand{\icon}[1]{
    \includegraphics[width=\unitsize, height=\unitsize]{icons/#1.png}%
}

\newcommand{\smallicon}[1]{
    \raisebox{-.2\height}{\includegraphics[width=\unitsize/2, height=\unitsize/2]{icons/#1.png}}%
}

% O T H E R
\newcommand{\divider}[2]{ % #1 > left node; #2 > right node
    \draw[dashed] ([xshift=\unitsize*0.7]#1.north) -- ([xshift=-\unitsize*0.8]#2.south);
}

\newcommand{\connect}[2]{ % #1 > source node; #2 > target node
    \begin{scope}[on background layer]
        \draw[->] (#1) -- (#2);
    \end{scope}
}

% tcolorbox
\tcbset{
    grid format/.style={% style for all grids
        arc=0.5pt,
        valign=center,
        halign=center,
        skin=enhanced,
        boxrule=0.5pt,
        boxsep=0.5pt,
        colframe=black!25!white,
        colback=white,
        frame empty,
    },
    page box/.style={% style for page frames
        middle=0mm,
        arc=1pt,
        boxrule=0.5pt,
        boxsep=0.5pt,
        colframe=black!25!white,
        colback=white
    },
    warning/.style={% style for warnings
        size=fbox,
        on line,
        colback=red!5!white,
        colframe=red!75!black
    },
    nodebox/.style={% style for tcboxes containing nodes/trees
        %show bounding box % debug borders around tcboxes (the trees)
    }
}
% END META

% HEADER
\fancyhf{}
\lhead{\thetitle \space -- \theversion \space -- \href{\thelicenseurl}{\thelicense}}
\chead{\hspace{8em}\tcbox[warning]{\theinfo}}
\rhead{\today}
\renewcommand{\headrulewidth}{0pt} %remove header underline
\lfoot{\theauthor \space -- \url{\theauthorsgithub}}
\cfoot{\thepage}
\rfoot{Numbers, Ratios \& Images from \url{\thewiki}}
% END HEADER

\begin{document}

\begin{tcolorbox}[height fill,
                  raster height=\boxheight,
                  page box]
	\section{Building Materials}
\begin{tcbraster}[raster columns=32,
	grid format]
	\tcbox[blankest,
	raster multicolumn=10,left=0mm,
	raster multirow=9]{
		\begin{tcbraster}[raster height=\boxheight,
			raster columns=1,
			grid format]
			\tcbox[nodebox, raster multicolumn=1, raster multirow=1]{ % Timber
				\begin{tikzpicture}
					\prodnode{(0,0)}{Wood}{1}{Lumberjack\%27s_Hut_(Old_World)}
					\prodnodec{(3,0)}{Timber}{1}{Sawmill_(Old_World)}{15}
					\connect{Wood}{Timber}
			\end{tikzpicture}}
			\tcbox[nodebox, raster multicolumn=1, raster multirow=1]{ % Bricks
				\begin{tikzpicture}
					\consnodelbld{(0,0)}{Clay}{1}{Clay_Pit_(Old_World)}{Clay}{\oldworld}{Artisans/25/1/0/7.5}
					\prodnodec{(3,0)}{Bricks}{2}{Brick_Factory_(Old_World)}{60}
					\connect{Clay}{Bricks}
			\end{tikzpicture}}
			\tcbox[nodebox, raster multicolumn=1, raster multirow=1]{% Wanza_Timber
				\begin{tikzpicture}
					\consnodec{(0,0)}{Wansa_Wood}{1}{Wanza_Woodcutter}{15}{Shepherds/50/1/0/0.5}
				\end{tikzpicture}}
			\tcbox[nodebox, raster multicolumn=1, raster multirow=2]{% Mud_bricks
				\begin{tikzpicture}
					\farmnode{(0,-1.5)}{Teff_Grass}{4}{Teff_Farm}{1×1}{144}
					\prodnodelbld{(0,0)}{Clay}{1}{Clay Collector}{Clay}{\enbesa}
					\prodnodec{(2,-0.75)}{Mud_bricks}{4}{Brick_Dry-House}{15}
					\connect{Teff_Grass}{Mud_bricks}
					\connect{Clay}{Mud_bricks}
				\end{tikzpicture}}
			\tcbox[nodebox, % Sails
			raster multicolumn=1,
			raster multirow=1]{
				\begin{tikzpicture}
					\farmnode{(0,0)}{Wool}{1}{Sheep_Farm}{3×3}{3}
					\prodnodec{(3,0)}{Sails}{1}{Sailmakers_(Old_World)}{30}
					\connect{Wool}{Sails}
			\end{tikzpicture}}
			\tcbox[nodebox, % Sails
			raster multicolumn=1,
			raster multirow=1]{
				\begin{tikzpicture}
					\farmnode{(0,2)}{Cotton}{2}{Cotton_Plantation}{1×1}{144}
					\prodnode{(2,2)}{Cotton_fabric}{1}{Cotton_Mill}
					\connect{Cotton}{Cotton_fabric}
					\prodnodec{(4,2)}{Sails}{1}{Sailmakers_(Old_World)}{30}
					\connect{Cotton_fabric}{Sails}
			\end{tikzpicture}}
			\tcbox[nodebox, raster multicolumn=1, raster multirow=2, top=0mm,=]{% Aluminium_Profiles
			\begin{tikzpicture}
				\prodnodelbld{(0,0)}{Charcoal_kiln}{3}{Charcoal_Kiln}{Coal}{\newworld}
				\prodnode{(0,-1.5)}{Bauxite}{1}{Bauxite_Mine}
				\prodnodec{(2,-0.75)}{Aluminium_Profiles}{9}{Aluminium_Smelter}{90}	
				\connect{Coal}{Aluminium_Profiles}
				\connect{Bauxite}{Aluminium_Profiles}
			\end{tikzpicture}}
		\end{tcbraster}}
	\tcbox[blankest,
	raster multicolumn=11, left=0mm, 
	raster multirow=9]{
		\begin{tcbraster}[raster height=\boxheight,
		raster columns=1,
		grid format]
		\tcbox[nodebox, raster multicolumn=1, raster multirow=2, top=-10mm]{% Steel Beams
		\begin{tikzpicture}
			\prodnode{(0.75,0)}{Iron}{1}{Iron_Mine}
			\prodnode{(0,-1.5)}{Charcoal_kiln}{2}{Charcoal_Kiln}
			\prodnode{(1.5,-1.5)}{Coal}{1}{Coal_Mine}
			\divider{Charcoal_kiln}{Coal}
			\prodnode{(3,-0.75)}{Steel}{2}{Furnace}
			\connect{Iron}{Steel}
			\connect{Coal}{Steel}
			\prodnodec{(5,-0.75)}{Steel_beams}{3}{Steelworks}{45}
			\connect{Steel}{Steel_beams}
		\end{tikzpicture}}
		\tcbox[nodebox, raster multicolumn=1, raster multirow=2, top=-10mm]{% Windows
		\begin{tikzpicture}
			\prodnode{(0,0)}{Quartz_sand}{2}{Sand_Mine}
			\prodnode{(2,0)}{Glass}{2}{Glassmakers}
			\connect{Quartz_sand}{Glass}
			
			\prodnode{(2,-1.5)}{Wood}{1}{Lumberjack\%27s_Hut_(Old_World)}
			\prodnodec{(4,-0.75)}{Windows}{4}{Window_Makers}{60}
			\connect{Glass}{Windows}
			\connect{Wood}{Windows}
		\end{tikzpicture}}
		\tcbox[nodebox, % Concrete
			raster multicolumn=3, top=-2mm,
			raster multirow=2]{
		\begin{tikzpicture}
			\prodnode{(0.75,0)}{Iron}{1}{Iron_Mine}
			\prodnode{(0,-1.5)}{Charcoal_kiln}{2}{Charcoal_Kiln}
			\prodnode{(1.5,-1.5)}{Coal}{1}{Coal_Mine}
			\divider{Charcoal_kiln}{Coal}
			\prodnode{(3,-0.75)}{Steel}{2}{Furnace}
			\connect{Iron}{Steel}
			\connect{Coal}{Steel}
			
			\prodnode{(3,-2.5)}{Cement}{2}{Limestone_Quarry}
			\prodnodec{(5,-1.5)}{Reinforced_concrete}{4}{Concrete_Factory}{60}
			\connect{Steel}{Reinforced_concrete}
			\connect{Cement}{Reinforced_concrete}
		\end{tikzpicture}}
		
		\tcbox[nodebox, % Helium
		raster multicolumn=1, top=6mm,
		raster multirow=2]{
			\begin{tikzpicture}
				\prodnode{(0,-0.75)}{Fish_Oil}{1}{Fish_Oil_Factory}
				\prodnode{(0,-2.25)}{Saltpeter}{4}{Saltpeter_Works}
				\prodnode{(2,-1.5)}{Industrial_Lubricant}{1}{Industrial_Oil_Press}
				\prodnodelbld{(2,0)}{Clay}{2}{Clay_Pit_}{Clay}{\newworld}
				\prodnodec{(4,-0.75)}{Helium}{1}{Helium_Extractor}{15}	
				\connect{Fish_Oil}{Industrial_Lubricant}
				\connect{Saltpeter}{Industrial_Lubricant}
				\connect{Clay}{Helium}
				\connect{Industrial_Lubricant}{Helium}
		\end{tikzpicture}}
	\end{tcbraster}}
   	\tcbox[blankest,
    raster multicolumn=11, left=0mm,
    raster multirow=9]{
    	\begin{tcbraster}[raster height=\boxheight,
    		raster columns=1,
    		grid format]
		\tcbox[nodebox, % Weapons
		raster multicolumn=1, top=-20mm,
		raster multirow=2]{
		\begin{tikzpicture}
			\prodnode{(0.75,0)}{Iron}{1}{Iron_Mine}
			\prodnode{(0,-1.5)}{Charcoal_kiln}{2}{Charcoal_Kiln}
			\prodnode{(1.5,-1.5)}{Coal}{1}{Coal_Mine}
			\divider{Charcoal_kiln}{Coal}
			\prodnode{(3,-0.75)}{Steel}{2}{Furnace}
			\connect{Iron}{Steel}
			\connect{Coal}{Steel}
			\prodnodec{(5,-0.75)}{Weapons}{6}{Weapon_Factory}{90}
			\connect{Steel}{Weapons}
		\end{tikzpicture}}
		\tcbox[nodebox, % Advanced Weapons
		raster multicolumn=1,top=-10mm,
		raster multirow=3]{
			\begin{tikzpicture}
				\farmnode{(0,0)}{Pigs}{4}{Pig_Farm}{2×3}{5}
				\prodnode{(2,0)}{Tallow}{4}{Rendering_Works}
				\connect{Pigs}{Tallow}
				
				\prodnode{(2,-1.5)}{Saltpeter}{8}{Saltpeter_Works}
				\consnode{(4.5,-0.75)}{Dynamite}{4}{Dynamite_Factory}{Technicians/166.67/3/0/0}
				\connect{Tallow}{Dynamite}
				\connect{Saltpeter}{Dynamite}
				
				\prodnode{(1,-3)}{Iron}{1}{Iron_Mine}
				\prodnode{(0.25,-4.5)}{Charcoal_kiln}{2}{Charcoal_Kiln}
				\prodnode{(1.75,-4.5)}{Coal}{1}{Coal_Mine}
				\divider{Charcoal_kiln}{Coal}
				\prodnode{(3.5,-3.75)}{Steel}{2}{Furnace}
				\connect{Iron}{Steel}
				\connect{Coal}{Steel}
				\prodnodefec{(6,-2.75)}{Advanced_weapons}{4}{Heavy_Weapons_Factory}{60}
				\connect{Dynamite}{Advanced_weapons}
				\connect{Steel}{Advanced_weapons}
		\end{tikzpicture}}
		\tcbox[nodebox, % Steam Engines
		raster multicolumn=1, top=10mm,
		raster multirow=3]{
			\begin{tikzpicture}
				\prodnode{(0.75,0)}{Iron}{1}{Iron_Mine}
				\prodnode{(0,-1.5)}{Charcoal_kiln}{2}{Charcoal_Kiln}
				\prodnode{(1.5,-1.5)}{Coal}{1}{Coal_Mine}
				\divider{Charcoal_kiln}{Coal}
				\prodnode{(3,-0.75)}{Steel}{2}{Furnace}
				\connect{Iron}{Steel}
				\connect{Coal}{Steel}
				
				\prodnode{(0.75,-3)}{Zinc}{2}{Zinc_Mine}
				\prodnode{(0.75,-4.5)}{Copper}{2}{Copper_Mine}
				\prodnode{(3, -3.75)}{Brass}{4}{Brass_Smeltery}
				\connect{Zinc}{Brass}
				\connect{Copper}{Brass}
				\prodnodefec{(5, -2.25)}{Steam_motors}{3}{Motor_Assembly_Line}{45}
				\connect{Steel}{Steam_motors}
				\connect{Brass}{Steam_motors}
		\end{tikzpicture}}
	\end{tcbraster}}
\end{tcbraster}
\end{tcolorbox}

\pagebreak

\begin{landscape} % SIDEWAYS PAGE
\begin{tcolorbox}[raster height=\boxheight,
                  height=\paperwidth-2cm, % subtract margins manually
                  page box]

% START Farmer Consumables
\section{Farmer Consumables}
\begin{tcbraster}[raster columns=5,
                  grid format]
    \tcbox[nodebox, % Fish
           raster multicolumn=1,
           raster multirow=2]{
        \begin{tikzpicture}
            \consnode{(0,0)}{Fish}{1}{Fishery}{Farmers/80/3/0/1,Workers/40/3/0/2}
        \end{tikzpicture}}
    \tcbox[nodebox, %Schnapps
           raster multicolumn=2,
           raster multirow=2]{
        \begin{tikzpicture}
            \farmnode{(0,0)}{Potato}{1}{Potato_Farm}{1×1}{72}
            \consnode{(3,0)}{Schnapps}{1}{Schnapps_Distillery}{Farmers/60/0/8/3,Workers/30/0/4/6,Explorers/133.3/0/12/3,Technicians/66.7/0/6/7,Jornaleros/30/4/0/5}
            \connect{Potato}{Schnapps}
        \end{tikzpicture}}
    \tcbox[nodebox, % Work Clothes
           raster multicolumn=2,
           raster multirow=2]{
        \begin{tikzpicture}
            \farmnodec{(0,0)}{Wool}{1}{Sheep_Farm}{3×3}{3}{Artisans/32.33/1/0/0}
            \consnode{(3.25,0)}{Work_clothes}{1}{Framework_Knitters}{Farmers/65/2/0/3,Workers/32.5/2/0/6,Jornaleros/133.33/2/0/1.1}
            \connect{Wool}{Work_clothes}
        \end{tikzpicture}}
\end{tcbraster}
% END Farmer Consumables

\tcbline
\section{Worker Consumables}
\begin{tcbraster}[raster columns=7,
	grid format]
	\tcbox[nodebox, % Bread
	raster multicolumn=4,left=0mm,
	raster multirow=2]{
		\begin{tikzpicture}
			\farmnode{(0,0)}{Grain}{2}{Grain_Farm}{1×1}{144}
			\consnode{(2.75,0)}{Flour}{1}{Flour_Mill}{Farmers/200/1/0/0.8}
			\consnode{(6.5,0)}{Bread}{2}{Bakery}{Workers/55/3/0/4,Artisans/27.5/6/0/12,Tourists/2/55/0/275,Explorers/100/3/0/0}
			\connect{Grain}{Flour}
			\connect{Flour}{Bread}
	\end{tikzpicture}}
	\tcbox[nodebox, % Beer
	raster multicolumn=3, valign=bottom, left=-1mm,
	raster multirow=2]{
		\begin{tikzpicture}
			\farmnodec{(0,0)}{Grain}{2}{Grain_Farm}{1×1}{144}{Shepherds/2000/1/0/0}
			\prodnode{(2.25,0)}{Malt}{1}{Malthouse}
			\connect{Grain}{Malt}
			\farmnode{(2.25,-1.5)}{Hops}{3}{Hop_Farm}{1×1}{96}
			\consnode{(4.5,-0.75)}{Beer}{2}{Brewery}{Workers/65/0/3/10,Artisans/32.5/0/3/30,Obreros/37.5/0/4/6}
			\connect{Malt}{Beer}
			\connect{Hops}{Beer}
	\end{tikzpicture}}
	\tcbox[nodebox, % Sausages
	raster multicolumn=2, 
	raster multirow=1]{
		\begin{tikzpicture}
			\farmnode{(0,0)}{Pigs}{1}{Pig_Farm}{2×3}{5}
			\consnode{(2,0)}{Sausages}{1}{Slaughterhouse}{Workers/50/3/0/4,Artisans/25/6/0/12}
			\connect{Pigs}{Sausages}
	\end{tikzpicture}}
	\tcbox[nodebox, % Soap
	raster multicolumn=5, top=10mm,
	raster multirow=1]{
		\begin{tikzpicture}
			\farmnode{(0,0)}{Pigs}{2}{Pig_Farm}{2×3}{5}
			\consnode{(2.5,0)}{Tallow}{2}{Rendering_Works}{Explorers/50/5/5/0}
			\consnode{(6,0)}{Soap}{1}{Soap_Factory}{Workers/240/2/0/4,Artisans/120/4/0/12,Engineers/25/5/0/0}
			\connect{Pigs}{Tallow}
			\connect{Tallow}{Soap}
	\end{tikzpicture}}
\end{tcbraster}
\tcbline
\section{Artisan Consumables}
\begin{tcbraster}[raster columns=\portraitraster,
	grid format]
	\tcbox[nodebox, % Canned Food
	raster multicolumn=6, left=-5mm,
	raster multirow=3]{
		\begin{tikzpicture}
			\consnodelbld{(0,-2)}{Beef}{8}{Cattle_Farm}{Beef_o}{\oldworld}{Workers/40/2/0/5}
			\consnodelbld{(1.5,-2)}{Beef}{4}{Cattle_Farm}{Beef_n}{\newworld}{Workers/20/0/0/0}
			\divider{Beef_o}{Beef_n}
			
			\farmnode{(0.75,-0)}{Red_peppers}{8}{Red_Pepper_Farm}{1×1}{108}
			\prodnode{(3.5,-0.75)}{Goulash}{8}{Artisanal_Kitchen}
			\connect{Beef_n}{Goulash}
			\connect{Red_peppers}{Goulash}
			
			\prodnode{(3.5,-2.5)}{Iron}{1}{Iron_Mine}
			
			\consnode{(7,-1.5)}{Canned_food}{6}{Cannery}{Artisans/65/4/0/6,Engineers/32.5/12/0/16,Technicians/55.55/3/0/10,Scholars/11.11/19/0/1,Shepherds/66.66/3/0/2.2}
			\connect{Goulash}{Canned_food}
			\connect{Iron}{Canned_food}
	\end{tikzpicture}}

	\tcbox[nodebox, % Fur Coats
		raster multicolumn=3, left=0mm, top=6mm,
		raster multirow=2]{
	\begin{tikzpicture}
		\farmnode{(0,0)}{Cotton}{4}{Cotton_Plantation}{1×1}{144}
		\consnode{(2,0)}{Cotton_fabric}{2}{Cotton_Mill}{Elders/50/4/0/0}
		\connect{Cotton}{Cotton_fabric}
		
		\consnodelbld{(1.75,-2.5)}{Furs}{4}{Hunting_Cabin}{Furs_o}{\oldworld}{Investors/62.5/3/0/3}
		\consnodelbld{(0.0,-2.5)}{Furs}{1}{Hunting_Cabin}{Furs_a}{\arctic}{Investors/250/0/0/0}
		\divider{Furs_a}{Furs_o}
		
		\consnode{(5.25,-1.25)}{Fur_Coats}{2}{Fur_Dealer}{Artisans/75/2/0/18,Engineers/37.5/6/0/48,Tourists/2/0/3/600}
		\connect{Cotton_fabric}{Fur_Coats}
		\connect{Furs_o}{Fur_Coats}
	\end{tikzpicture}}
	\tcbox[nodebox, % Sewing Machines
		raster multicolumn=3, left=3mm, top=5mm,
		raster multirow=2]{
	\begin{tikzpicture}
		\prodnode{(0.75,0)}{Iron}{1}{Iron_Mine}
		\prodnode{(0,-1.5)}{Charcoal_kiln}{2}{Charcoal_Kiln}
		\prodnode{(1.5,-1.5)}{Coal}{1}{Coal_Mine}
		\divider{Charcoal_kiln}{Coal}
		\prodnode{(3,-0.75)}{Steel}{2}{Furnace}
		\connect{Iron}{Steel}
		\connect{Coal}{Steel}
		
		\prodnode{(3,-2.5)}{Wood}{1}{Lumberjack\%27s_Hut_(Old_World)}
		
		\consnode{(5.5,-1.5)}{Sewing_machines}{2}{Sewing_Machine_Factory}{Artisans/70/2/0/12,Engineers/35/6/0/32,Obreros/80/2/0/5,Artistas/80/5/0/16,Elders/111.11/2/0/6.4}
		\connect{Steel}{Sewing_machines}
		\connect{Wood}{Sewing_machines}
	\end{tikzpicture}}
\end{tcbraster}
\tcbline
\include{Bright_harvest.tex}
\end{tcolorbox}
\end{landscape}

\pagebreak

\begin{landscape} % SIDEWAYS PAGE
	\begin{tcolorbox}[raster height=\boxheight,
	height=\paperwidth-2cm, % subtract margins manually
	page box]
		\section{Engineer Consumables}
\begin{tcbraster}[raster columns=\portraitraster,
	grid format]
	% Glasses
	\begin{tcolorbox}[nodebox, raster multicolumn=3, raster multirow=3,halign lower=right, size=minimal,lower separated=false, valign=center, halign=center]
		\begin{tikzpicture}
			\prodnode{(0,0)}{Quartz_sand}{1}{Sand_Mine}
			\prodnode{(2,0)}{Glass}{1}{Glassmakers}
			\connect{Quartz_sand}{Glass}
			
			\prodnode{(0,-1.5)}{Zinc}{1}{Zinc_Mine}
			\prodnode{(0,-3)}{Copper}{1}{Copper_Mine}
			\prodnode{(2, -2.25)}{Brass}{2}{Brass_Smeltery}
			\connect{Zinc}{Brass}
			\connect{Copper}{Brass}
			
			\consnode{(4.75,-1.5)}{Glasses}{3}{Spectacle_Factory}{Engineers/75/4/0/20,Investors/37.5/16/0/50,Elders/92.5/0/3/6.9,Artistas/16.66/22/0/90,Obreros/41.66/3/0/2}
			\connect{Glass}{Glasses}
			\connect{Brass}{Glasses}
		\end{tikzpicture}
	\end{tcolorbox}
	\tcbox[nodebox, % Penny Farthings
	raster multicolumn=3, valign=top,left=0mm,
	raster multirow=3]{
		\begin{tikzpicture}
			\prodnode{(0.75,0)}{Iron}{1}{Iron_Mine}
			\prodnode{(0,-1.5)}{Charcoal_kiln}{2}{Charcoal_Kiln}
			\prodnode{(1.5,-1.5)}{Coal}{1}{Coal_Mine}
			\divider{Charcoal_kiln}{Coal}
			\prodnode{(3,-0.75)}{Steel}{2}{Furnace}
			\connect{Iron}{Steel}
			\connect{Coal}{Steel}
			
			\farmnode{(3,-2.5)}{Caoutchouc}{4}{Caoutchouc_Plantation}{1×1}{144}
			
			\consnodefe{(5.5,-1.5)}{High_wheeler}{1}{Bicycle_Factory}{Engineers/160/0/5/28,Investors/80/0/4/70,Workers/500/4/0/3.2}
			\connect{Steel}{High_wheeler}
			\connect{Caoutchouc}{High_wheeler}
	\end{tikzpicture}}
	\tcbox[nodebox, % Pocket Watches
	raster multicolumn=3,
	raster multirow=3]{
		\begin{tikzpicture}
			\prodnodelbld{(0,0)}{Gold_Ore}{10}{Gold_Mine}{Gold_Ore_n}{\newworld}
			\prodnodelbld{(1.5,0)}{Gold_Ore}{4}{Gold_Mine}{Gold_Ore_a}{\arctic}
			\divider{Gold_Ore_n}{Gold_Ore_a}
			
			\prodnode{(0,-1.5)}{Charcoal_kiln}{2}{Charcoal_Kiln}
			\prodnode{(1.5,-1.5)}{Coal}{1}{Coal_Mine}
			\divider{Charcoal_kiln}{Coal}
			
			\prodnode{(3, -0.75)}{Gold}{4}{Goldsmiths}
			\connect{Gold_Ore_a}{Gold}
			\connect{Coal}{Gold}
			
			\prodnode{(0.75,-3)}{Quartz_sand}{2}{Sand_Mine}
			\prodnode{(3,-3)}{Glass}{2}{Glassmakers}
			\connect{Quartz_sand}{Glass}
			
			\consnodefe{(5.25,-1.5)}{Pocket_watch}{3}{Clockmakers}{Engineers/170/0/3/36,Investors/85/0/3/90}
			\connect{Gold}{Pocket_watch}
			\connect{Glass}{Pocket_watch}
	\end{tikzpicture}}
	\tcbox[blankest,
		raster multicolumn=3,
		raster multirow=3]{
	    \begin{tcbraster}[raster height=\boxheight,
			raster columns=3,
			grid format]
			\tcbox[nodebox, % Light Bulbs
				raster multicolumn=3,left=0mm,
				raster multirow=2]{
			\begin{tikzpicture}
				\prodnode{(0,0)}{Charcoal_kiln}{2}{Charcoal_Kiln}
				\prodnode{(1.5,0)}{Coal}{1}{Coal_Mine}
				\divider{Charcoal_kiln}{Coal}
				\prodnode{(3, 0)}{Carbon_filament}{4}{Filament_Factory}
				\connect{Coal}{Carbon_filament}
				
				\prodnode{(0.75,-1.5)}{Quartz_sand}{2}{Sand_Mine}
				\prodnode{(3,-1.5)}{Glass}{2}{Glassmakers}
				\connect{Quartz_sand}{Glass}
				
				\consnode{(5.5,-0.75)}{Light_bulb}{4}{Light_Bulb_Factory}{Engineers/80/2/0/28,Investors/40/8/0/70,Artistas/50/4/0/4.8}
				\connect{Carbon_filament}{Light_bulb}
				\connect{Glass}{Light_bulb}
			\end{tikzpicture}}
			\tcbox[nodebox, % Oil_Power_Plant
				raster multicolumn=3,
				raster multirow=2]{
			\begin{tikzpicture}
				\prodnode{(0,0)}{Oilwell}{3}{Oil_Well}
				\consnode{(2.5,0)}{Oil_Power_Plant}{1}{Oil_Power_Plant}{Engineers/x/2/0/0,Investors/x/8/0/0,Artistas/x/20/0/54,Scholars/x/12/0/1}
				\connect{Oilwell}{Oil_Power_Plant}
			\end{tikzpicture}}
		\end{tcbraster}}
\end{tcbraster}
		\tcbline
		\include{investors.tex}
	\end{tcolorbox}
\end{landscape}
\pagebreak

\begin{landscape} % SIDEWAYS PAGE
	\begin{tcolorbox}[raster height=\boxheight,
	height=\paperwidth-2cm, % subtract margins manually
	page box]
		\section{Jornalero Consumables}
\begin{tcbraster}[raster columns=\portraitraster,
	grid format]
	\tcbox[nodebox, % Fried Plantains
	raster multicolumn=2,
	raster multirow=2]{
		\begin{tikzpicture}
			\farmnode{(0,0)}{Plantains}{1}{Plantain_Plantation}{1×1}{128}
			\prodnode{(0,-1.5)}{Fish_Oil}{1}{Fish_Oil_Factory}
			\consnode{(2.5,-0.75)}{Fried_plantains}{1}{Fried_Plantain_Kitchen}{Jornaleros/70/3/0/2,Obreros/35/3/0/5}
			\connect{Plantains}{Fried_plantains}
			\connect{Fish_Oil}{Fried_plantains}
	\end{tikzpicture}}
	\tcbox[nodebox, % Rum
	raster multicolumn=2,
	raster multirow=2]{
		\begin{tikzpicture}
			\farmnode{(0,0)}{Sugar_cane}{2}{Sugar_Cane_Plantation}{1×1}{128}
			\prodnode{(0,-1.5)}{Wood}{1}{Lumberjack\%27s_Hut_(New_World)}
			\consnode{(3,-0.75)}{Rum}{2}{Rum_Distillery}{Artisans/35/0/4/15,Engineers/17.5/0/4/40,Jornaleros/140/0/6/2,Obreros/70/0/3/5,Scholars/84/0/2/2,Workers/100/2/0/2.4,Technicians/50/3/10/2.4}
			\connect{Sugar_cane}{Rum}
			\connect{Wood}{Rum}
	\end{tikzpicture}}
	\tcbox[nodebox, % Ponchos
	raster multicolumn=2,
	raster multirow=2]{
		\begin{tikzpicture}
			\farmnode{(0,0)}{Alpaca_wool}{1}{Alpaca_Farm}{3×3}{4}
			\consnode{(2.5,0)}{Poncho}{1}{Poncho_Darner}{Jornaleros/80/2/0/2,Obreros/40/2/0/5,Shepherds/62.5/2/0/0.4}
			\connect{Alpaca_wool}{Poncho}
	\end{tikzpicture}}
\end{tcbraster}
		\tcbline
		\section{Obrero Consumables}
\begin{tcbraster}[raster columns=8,
	grid format]
	\tcbox[nodebox, % Tortillas
	raster multicolumn=4, left=0mm, top=3mm,
	raster multirow=2]{
		\begin{tikzpicture}
			\farmnode{(0.75,0)}{Corn}{2}{Corn_Farm}{1×1}{168}
			
			\farmnodelbld{(0,-2)}{Beef}{4}{Cattle_Farm}{3×4}{4}{Beef_o}{\oldworld}
			\farmnodelbld{(1.5,-2)}{Beef}{2}{Cattle_Farm}{3×4}{6}{Beef_n}{\newworld}
			\divider{Beef_o}{Beef_n}
			
			\consnode{(4,-1)}{Tortilla}{1}{Tortilla_Maker}{Obreros/70/4/0/2,Artistas/70/5/0/8}
			\connect{Corn}{Tortilla}
			\connect{Beef_n}{Tortilla}
	\end{tikzpicture}}
	\tcbox[nodebox, % Coffee
		raster multicolumn=4, left=-8mm, top=-7mm,
		raster multirow=2]{
	\begin{tikzpicture}
		\farmnode{(0,0)}{Coffee_beans}{2}{Coffee_Plantation}{1×1}{168}
		\consnode{(3.25,0)}{Coffee}{1}{Coffee_Roaster}{Obreros/170/2/0/2,Artistas/170/5/0/8,Engineers/42.5/2/0/16,Investors/21.25/8/0/40,Technicians/83.3/0/6/12}
		\connect{Coffee_beans}{Coffee}
	\end{tikzpicture}}
	\tcbox[nodebox, % Cigars
	raster multicolumn=4, left=0mm, 
	raster multirow=2]{
		\begin{tikzpicture}
			\prodnode{(0,0)}{Wood}{1}{Lumberjack\%27s_Hut_(New_World)}
			\prodnode{(1.75,0)}{Wood_veneers}{4}{Marquetry_Workshop_(New_World)}
			\connect{Wood}{Wood_veneers}
			\farmnode{(1.5,-1.5)}{Tobacco}{8}{Tobacco_Plantation}{1×1}{192}
			\consnode{(4.25,-0.75)}{Cigars}{2}{Cigar_Factory}{Obreros/180/0/2/7,Artistas/180/0/3/14,Investors/90/2/0/25}
			\connect{Wood_veneers}{Cigars}
			\connect{Tobacco}{Cigars}
	\end{tikzpicture}}
	\tcbox[nodebox, % Chocolate
	raster multicolumn=4, left=0mm,
	raster multirow=2]{
	\begin{tikzpicture}
		\farmnode{(0,-1.75)}{Sugar_cane}{1}{Sugar_Cane_Plantation}{1×1}{128}
		\consnode{(2.75,-2)}{Sugar}{1}{Sugar_Refinery}{Farmers/166.6/1/0/1.9}
		\connect{Sugar_cane}{Sugar}
		
		\farmnode{(2.75,0)}{Cocoa}{2}{Cocoa_Plantation}{1×1}{128}
		
		\consnode{(5.75,-1)}{Chocolate}{1}{Chocolate_Factory}{Investors/37.5/2/0/25,Engineers/100/5/0/0}
		\connect{Chocolate}{Sugar}
		\connect{Cocoa}{Chocolate}
	\end{tikzpicture}}
	\tcbox[nodebox, % Bowler Hats
	raster multicolumn=5, left=0mm, top=5mm,
	raster multirow=2]{
	\begin{tikzpicture}
		\farmnode{(0,0)}{Cotton}{2}{Cotton_Plantation}{1×1}{144}
		\consnode{(2.5,0)}{Cotton_fabric}{1}{Cotton_Mill}{Elders/33.33/3/0/3.4}
		\connect{Cotton}{Cotton_fabric}
		
		\farmnode{(0,-2)}{Alpaca_wool}{1}{Alpaca_Farm}{3×3}{4}
		\consnode{(2.5,-2)}{Felt}{1}{Felt_Producer}{Jornaleros/88.8/1/0/1.4}
		\connect{Alpaca_wool}{Felt}
		
		\consnode{(6.5,-1)}{Bowler_hats}{1}{Bombin_Weaver}{Obreros/75/2/0/2,Artistas/75/5/0/4,Scholars/41.7/0/3/2,Tourists/10/15/0/90}
		\connect{Cotton_fabric}{Bowler_hats}
		\connect{Felt}{Bowler_hats}
	\end{tikzpicture}}

\end{tcbraster}
		\tcbline
		\section{Hacienda Consumables}
\begin{tcbraster}[raster columns=3,
	grid format]
	\tcbox[nodebox, % empty
		raster multicolumn=1,
		raster multirow=2]{}
	\tcbox[nodebox, % Schnapps
		raster multicolumn=1,
		raster multirow=2]{
		\begin{tikzpicture}
			\hacbrewnode{(0,0)}{Potato}{1}{1×1}{64}
			\hacbrewnode{(2,0)}{Schnapps}{2}
			\connect{Potato}{Schnapps}
		\end{tikzpicture}}
	\tcbox[nodebox, % Hot_sauce
		raster multicolumn=1, left=0mm, top=-7mm,
		raster multirow=2]{
	\begin{tikzpicture}
			\hacbrewnode{(0,0)}{Spices}{1}{1×1}{64}
			\hacbrewnodecons{(2.75,0)}{Hot_sauce}{2}{Jornaleros/10/4/0/2,Obreros/8.75/4/0/4,Workers/50/1/0/0,Explorers/25/3/2/0,Shepherds/100/2/0/1.5}
			\connect{Spices}{Hot_sauce}
	\end{tikzpicture}}
	\tcbox[nodebox, % Atole
		raster multicolumn=1, left=0mm,
		raster multirow=2]{
	\begin{tikzpicture}
		\hacfarmnode{(0,0)}{Sugar_cane}{1}{1×1}{64}
		\hacfarmnode{(0,-1.5)}{Corn}{2}{1×1}{64}
		\hacbrewnodecons{(2.75,-0.75)}{Atole}{2}{Obreros/8.74/4/0/16}
		\connect{Sugar_cane}{Atole}
		\connect{Corn}{Atole}
	\end{tikzpicture}}
	\tcbox[nodebox, % Rum
		raster multicolumn=1,
		raster multirow=2]{
	\begin{tikzpicture}
		\hacfarmnode{(0,0)}{Sugar_cane}{2}{1×1}{64}
		\prodnode{(0,-1.5)}{Wood}{1}{Lumberjack\%27s_Hut_(New_World)}
		\hacbrewnode{(2,-0.75)}{Rum}{2}
		\connect{Sugar_cane}{Rum}
		\connect{Wood}{Rum}
	\end{tikzpicture}}
	\tcbox[nodebox, % Atole
		raster multicolumn=1,
		raster multirow=2]{
	\begin{tikzpicture}
		\hacfarmnode{(0,0)}{Grain}{1}{1×1}{64}
		\hacfarmnode{(0,-1.5)}{Corn}{1}{1×1}{64}
		\hacbrewnode{(2,-0.75)}{Beer}{1}
		\connect{Grain}{Beer}
		\connect{Corn}{Beer}
	\end{tikzpicture}}
	\tcbox[nodebox, % Fertiliser_ratio
		raster multicolumn=1,
		raster multirow=1]{
	\begin{tikzpicture}
		\prodnode{(0,0)}{Fertiliser}{1}{Hacienda_Fertiliser_Works}
		\prodnode{(2,0)}{Silo}{10}{Fertiliser_Silo}
		\connect{Fertiliser}{Silo}
	\end{tikzpicture}}
	\tcbox[nodebox, % Fertiliser_ratio
		raster multicolumn=1,
		raster multirow=1]{
	\begin{tikzpicture}
		\prodnode{(2,0)}{Sugar_cane}{2.66}{Production_buildings}
		\prodnodelbld{(0,0)}{Sugar_cane}{1}{Production_buildings}{Silo}{\smallicon{Fertiliser}}
		\connect{Sugar_cane}{Silo}
	\end{tikzpicture}}
	\tcbox[nodebox, % Fertiliser_Works
		raster multicolumn=1,
		raster multirow=1]{
	\begin{tikzpicture}
		\prodnode{(0,0)}{Dung}{1}{Production_buildings}
		\prodnodec{(2,0)}{Fertiliser}{1}{Hacienda_Fertiliser_Works}{30}
		\connect{Dung}{Fertiliser}
	\end{tikzpicture}}
\end{tcbraster}

	\end{tcolorbox}
\end{landscape}
\pagebreak

\begin{landscape} % SIDEWAYS PAGE
	\begin{tcolorbox}[raster height=\boxheight,
		height=\paperwidth-2cm, % subtract margins manually
		page box]
		\section{Explorer Consumables}
\begin{tcbraster}[raster columns=4,
	grid format]
	\tcbox[nodebox, % Heater
		raster multicolumn=2,
		raster multirow=1]{
	\begin{tikzpicture}
		\prodnodelbld{(1.5,0)}{Charcoal_kiln}{1}{Charcoal_Kiln}{Coal_a}{\arctic}
		\prodnodelbld{(0,0)}{Charcoal_kiln}{1}{Charcoal_Kiln}{Coal_o}{\oldworld}
		\divider{Coal_o}{Coal_a}
		\consnode{(3.5,0)}{Heater}{1}{Heater}{Explorers/x/0/32/0,Technicians/x/0/32/0}
		\connect{Coal_a}{Heater}
	\end{tikzpicture}}
	\tcbox[nodebox, % Pemmican
		raster multicolumn=2, left=17mm,
		raster multirow=2]{
	\begin{tikzpicture}
		\prodnode{(0,0)}{Whale_Oil}{1}{Whaling Station}
		\prodnode{(0,-1.5)}{Caribou_Meat}{1}{Caribou Hunting Cabin}
		\consnode{(2.75,-0.75)}{Pemmican}{1}{Pemmican Cookhouse}{Explorers/83.3/3/0/4,Technicians/41.6/3/0/8}
		\connect{Whale_Oil}{Pemmican}
		\connect{Caribou_Meat}{Pemmican}
	\end{tikzpicture}}
	\tcbox[nodebox, % Sleeping Bags
	raster multicolumn=2,
	raster multirow=2]{
		\begin{tikzpicture}
			\prodnode{(0.25,0)}{Seal_Skin}{1}{Seal Hunting Docks}
			\farmnodec{(0,-1.5)}{Goose_Feathers}{4}{Goose Farm}{2x3}{5}{Elders/533.3/3/0/0}
			\consnode{(3.25,-0.75)}{Sleeping_Bags}{2}{Sleeping Bag Factory}{Explorers/111.1/0/16/5,Technicians/55.55/0/8/10}
			\connect{Seal_Skin}{Sleeping_Bags}
			\connect{Goose_Feathers}{Sleeping_Bags}
	\end{tikzpicture}}
	\tcbox[nodebox, % Oil Lamps
	raster multicolumn=3,
	raster multirow=2]{
		\begin{tikzpicture}
			\prodnode{(0,0)}{Copper}{1}{Copper_Mine}
			\prodnode{(0,-1.5)}{Zinc}{1}{Zinc_Mine}
			\prodnode{(2,-0.75)}{Brass}{2}{Brass_Smeltery}
			\connect{Copper}{Brass}
			\connect{Zinc}{Brass}
			
			\prodnode{(2,-2.5)}{Whale_Oil}{2}{Whaling_Station}
			
			\consnode{(4.75,-1.5)}{Oil_Lamps}{2}{Oil_Lamp_Factory}{Explorers/55.55/3/0/7,Technicians/27.25/3/0/14}
			\connect{Brass}{Oil_Lamps}
			\connect{Whale_Oil}{Oil_Lamps}
		\end{tikzpicture}
		\vspace{0.5\unitsize}}
\end{tcbraster}
		\tcbline
		\section{Technician Consumables}
\begin{tcbraster}[raster columns=6,
	grid format]
	\tcbox[nodebox, % Parkas
	raster multicolumn=3,
	raster multirow=2]{
		\begin{tikzpicture}
			\consnode{(0,-1.5)}{Bear_Skin}{3}{Bear_Hunting_Cabin}{Investors/66.6/10/0/0}
			\prodnode{(0,0)}{Seal_Skin}{1}{Seal_Hunting_Docks}
			\consnode{(3,-0.75)}{Parkas}{3}{Parka_Factory}{Technicians/27.7/0/8/9}
			\connect{Bear_Skin}{Parkas}
			\connect{Seal_Skin}{Parkas}
	\end{tikzpicture}}
	\tcbox[nodebox, % Husky Sleds
	raster multicolumn=3,
	raster multirow=2]{
		\begin{tikzpicture}
			\prodnode{(0,0)}{Seal_Skin}{2}{Seal_Hunting_Docks}
			\prodnode{(0,-1.5)}{Wood}{1}{Lumberjack\%27s_Hut}
			\prodnode{(2,-0.75)}{Sleds}{4}{Sled_Frame_Factory}
			\connect{Seal_Skin}{Sleds}
			\connect{Wood}{Sleds}
			\farmnode{(2,-2.5)}{Huskies}{8}{Husky_Farm}{3x4}{4}
			\consnode{(4.75,-1.5)}{Husky_Sleds}{4}{Husky_Sled_Factory}{Technicians/55.55/4/0/16}
			\connect{Sleds}{Husky_Sleds}
			\connect{Huskies}{Husky_Sleds}
		\end{tikzpicture}
		\vspace{0.5\unitsize}}
	\tcbox[nodebox, % Gas_power_plant
	raster multicolumn=2,
	raster multirow=1]{
		\begin{tikzpicture}
			\prodnode{(0,0)}{Gas}{6}{Gas_Pumps}
			\prodnode{(2,0)}{Gas_power_plant}{1}{Gas-Fired_Power_Plant}
			\connect{Gas}{Gas_power_plant}
	\end{tikzpicture}}
	\tcbox[nodebox, % Gold ratio
		raster multicolumn=2,
		raster multirow=1]{
	\begin{tikzpicture}
		\prodnodelbld{(0,0)}{Gold_Ore}{2.5}{Gold_Mine}{Gold_Ore_n}{\newworld}
		\prodnodelbld{(2,0)}{Gold_Ore}{1}{Gold_Mine}{Gold_Ore_a}{\arctic}
		\connect{Gold_Ore_n}{Gold_Ore_a}
	\end{tikzpicture}}
	\tcbox[nodebox, % Fur Ration
		raster multicolumn=2,
		raster multirow=1]{
	\begin{tikzpicture}
		\prodnodelbld{(0,0)}{Furs}{4}{Hunting_Cabin}{Furs_o}{\oldworld}
		\prodnodelbld{(2,0)}{Furs}{1}{Hunting_Cabin}{Furs_a}{\arctic}
		\connect{Furs_o}{Furs_a}
\end{tikzpicture}}
\end{tcbraster}
		\tcbline
		\section{AirMail}
\begin{tcbraster}[raster columns=2,
	grid format]
	\tcbox[nodebox, % regular Mail
	raster multicolumn=2, top=-5mm,
	raster multirow=5]{
	\begin{tikzpicture}
		\consnode{(0,-0.75)}{Local_Mail}{1}{Local_Mail}{Farmers/62.5/1/0/1,Workers/30.77/2/0/4,Artisans/21.74/5/0/6,Engineers/16/3/0/16,Investors/12.9/8/0/25,Jornaleros/62.5/1/0/2,Obreros/21.74/4/0/2,Artistas/13.33/4/0/6,Explorers/66.66/1/0/0/0,Technicians/33.3/3/0/0/0,Scholars/5.56/10/0/0}	
		\consnode{(4.5,-0.75)}{Regional_Mail}{1}{Regional_Mail}{Farmers/125/1/0/3,Workers/125/3/0/6,Artisans/90.91/6/0/12,Engineers/66.67/7/0/48,Investors/52.63/12/0/70,Jornaleros/285.71/3/0/2.5,Obreros/90.91/5/0/4,Artistas/52.63/7/0/12,Explorers/200/2/0/0/0,Technicians/100/6/0/0/0,Scholars/11.11/15/0/0}	
		\consnode{(9,-0.75)}{Overseas_Mail}{1}{Overseas_Mailn}{Farmers/333.33/3/0/6,Workers/133.33/5/0/12,Artisans/100/14/0/18,Engineers/83.33/15/0/96,Investors/66.67/25/0/125,Jornaleros/333.33/5/0/5,Obreros/100/7/0/8,Artistas/66.67/12/0/22,Explorers/200/4/0/0/0,Technicians/100/10/0/0/0,Scholars/11.11/25/0/0}	
		\connect{Local_Mail}{Regional_Mail}
		\connect{Regional_Mail}{Overseas_Mail}
	\end{tikzpicture}}
	\tcbox[nodebox, % Special_Mail
	raster multicolumn=1, top=-5mm,
	raster multirow=1]{
	\begin{tikzpicture}
		\prodnodelbld{(0,-0.75)}{Local_Mail}{1}{Local_Mail}{New_World_Mail}{\newworld}
		\consnode{(2.25,-0.75)}{New_World_Reports}{1}{New_World_Reports}{Scholars/25/5/0/4.8}		
		\connect{New_World_Mail}{New_World_Reports}
	\end{tikzpicture}}
	\tcbox[nodebox, % Special_Mail
		raster multicolumn=1, top=-5mm,
		raster multirow=1]{
	\begin{tikzpicture}	
		\prodnodelbld{(0,-0.75)}{Local_Mail}{1}{Local_Mail}{Arctic_Mail}{\arctic}
		\consnode{(2.25,-0.75)}{Arctic_Reports}{1}{Arctic_Reports}{Scholars/25/5/0/12}	
		\connect{Arctic_Mail}{Arctic_Reports}
	\end{tikzpicture}}
\end{tcbraster}
		\tcbline
	\end{tcolorbox}
\end{landscape}
\pagebreak

\begin{landscape} % SIDEWAYS PAGE
	\begin{tcolorbox}[raster height=\boxheight,
		height=\paperwidth-2cm, % subtract margins manually
		page box]
		\include{shepards.tex}
		\tcbline
		\include{elders.tex}
		\tcbline
		\section{Scholar Consumables}
\begin{tcbraster}[raster columns=\portraitraster,
	grid format]
	\tcbox[nodebox, %Leather Boots
	raster multicolumn=3,
	raster multirow=2]{
		\begin{tikzpicture}
			\farmnode{(0,0)}{Sanga_cow}{1}{Sanga_Farm}{4×4}{5}
			\consnode{(2,0)}{Leather_Boots}{1}{BootMakers}{Scholars/1000}
			\connect{Sanga_cow}{Leather_Boots}
	\end{tikzpicture}}
	
	\tcbox[nodebox, % Leather_Suits
	raster multicolumn=3,
	raster multirow=2]{
		\begin{tikzpicture}
			\farmnode{(0,0)}{Linseed}{2}{Linseed_Farm}{1×1}{96}
			\prodnode{(2,0)}{Linen}{1}{Linen_Mill}
			\connect{Linseed}{Linen}
			\farmnode{(0,-1.5)}{Cotton}{2}{Cotton_Plantation}{1×1}{144}
			\prodnode{(2,-1.5)}{Cotton_fabric}{1}{Cotton_Mill}
			\connect{Cotton}{Cotton_fabric}
			\consnode{(4,-0.75)}{Tailored_Suits}{2}{Tailor's_Shop}{Scholars/736}
			\connect{Linen}{Tailored_Suits}
			\connect{Cotton_fabric}{Tailored_Suits}
	\end{tikzpicture}}

	\tcbox[nodebox, % Telephones
	raster multicolumn=3,
	raster multirow=2]{
	\begin{tikzpicture}
		\prodnode{(0,0)}{Wood}{1}{Lumberjack\%27s_Hut_(Old_World)}
		\prodnode{(2,0)}{Wood_veneers}{4}{Marquetry_Workshop_(Old_World)}
		\connect{Wood}{Wood_veneers}
		\prodnode{(0,-1.5)}{Charcoal_kiln}{2}{Charcoal_Kiln}
		\prodnode{(1.5,-1.5)}{Coal}{1}{Coal_Mine}
		\divider{Charcoal_kiln}{Coal}
		\prodnode{(3, -1.5)}{Carbon_filament}{4}{Filament_Factory}
		\connect{Coal}{Carbon_filament}
		\consnodefe{(5,-0.75)}{Telephones}{3}{Telephone_Manufacturer}{Scholars/1777}
		\connect{Wood_veneers}{Telephones}
		\connect{Carbon_filament}{Telephones}
\end{tikzpicture}}
\end{tcbraster}

\tcbline


	\end{tcolorbox}
\end{landscape}
\pagebreak

\begin{landscape} % SIDEWAYS PAGE
	\begin{tcolorbox}[raster height=\boxheight,
		height=\paperwidth-2cm, % subtract margins manually
		page box]
		\section{Artistas Consumables}
\begin{tcbraster}[raster columns=12,	grid format]
	\tcbox[nodebox, %Soccer Balls
	raster multicolumn=4, left=0mm,
	raster multirow=2]{
		\begin{tikzpicture}
			\farmnode{(0,0)}{nandu_leather}{1}{Nandu_Farm}{4×4}{4}
			\farmnode{(0,-2)}{Caoutchouc}{2}{Caoutchouc_Plantation}{1×1}{144}
			\consnode{(2.75,-1)}{soccer_balls}{1}{Ball_Manufactory}{Artistas/80/3/0/8,Workers/200/3/0/3.6,Artisans/100/2/0/5.1,Jornaleros/100/2/0/1.2}
			\connect{nandu_leather}{soccer_balls}
			\connect{Caoutchouc}{soccer_balls}
	\end{tikzpicture}}
	\tcbox[nodebox, %ice_cream
		raster multicolumn=8, top=-5mm,left=10mm,
		raster multirow=3]{
	\begin{tikzpicture}
		\orchnodenew{(4,-1.5)}{Citrus}{3}
		\farmnodep{(4,-3)}{Milk}{1}{Cattle_Farm}{3×4}{6}{Beef}
		
		\farmnode{(0,-0.75)}{Sugar_cane}{1}{Sugar_Cane_Plantation}{1×1}{128}
		\prodnode{(2,-0.75)}{Sugar}{1}{Sugar_Refinery}
		\connect{Sugar_cane}{Sugar}
		\farmnode{(2,0.75)}{Cocoa}{2}{Cocoa_Plantation}{1×1}{128}
		\prodnode{(4,0)}{Chocolate}{3}{Chocolate_Factory}
		\connect{Sugar}{Chocolate}
		\connect{Cocoa}{Chocolate}
		
		\consnode{(7,-1.5)}{ice_cream}{6}{Ice_Cream_Factory}{Artistas/180.6/0/3/24,Engineers/50/6/0/8,Tourists/10/25/0/225}
		\connect{Milk}{ice_cream}
		\connect{Citrus}{ice_cream}
		\connect{Chocolate}{ice_cream}
	\end{tikzpicture}}

	\tcbox[nodebox, %mezcal
	raster multicolumn=6,left=0mm,
	raster multirow=3]{
	\begin{tikzpicture}
		\farmnode{(0,0)}{Sugar_cane}{1}{Sugar_Cane_Plantation}{1×1}{128}
		\prodnode{(2,0)}{Sugar}{1}{Sugar_Refinery}
		\connect{Sugar_cane}{Sugar}
		\orchnodenew{(2,-1.5)}{Citrus}{1}
		\farmnode{(2,-3)}{Herbs}{1}{Herb_Garden}{1×1}{128}
		
		\consnode{(5,-1.5)}{Mezcal}{1}{Mezcal_Bar}{Artistas/25/2/0/12,Engineers/133.33/3/0/8.8,Technicians/142/2/5/0,Tourists/11/20/0/60}
		\connect{Sugar}{Mezcal}
		\connect{Citrus}{Mezcal}
		\connect{Herbs}{Mezcal}
	\end{tikzpicture}}
	\tcbox[nodebox, %Perfumes
		raster multicolumn=6,
		raster multirow=3]{
	\begin{tikzpicture}
		\prodnode{(0,-0.75)}{Wood}{1}{Lumberjack\%27s_Hut_(Old_World)}
		\farmnode{(0,-2.25)}{Corn}{4}{Corn_Farm}{1×1}{168}
		\chemnodenew{(2,-1.5)}{Ethanol}{2}
		\connect{Wood}{Ethanol}
		\connect{Corn}{Ethanol}
		\orchnodenew{(2,0)}{Coconut_Oil}{2}
		\farmnodec{(2,-3.25)}{Orchid}{2}{Orchid_Farm}{1×1}{128}{Elders/40/3/0/2.8}
		
		\consnode{(5,-1.5)}{Perfumes}{4}{Perfume_Mixer}{Artistas/20/3/0/28,Artisans/133.33/4/0/9.6,Investors/80/7/0/9,Tourists/17.4/26/0/140}
		\connect{Perfumes}{Orchid}
		\connect{Ethanol}{Perfumes}
		\connect{Coconut_Oil}{Perfumes}
	\end{tikzpicture}}
	\tcbox[nodebox, %Sewing Machine
		raster multicolumn=6,
		raster multirow=3]{
	\begin{tikzpicture}
		\prodnodelbld{(0,-1.5)}{Charcoal_kiln}{6}{Charcoal_Kiln}{Coal}{\newworld}
		\prodnode{(0,-3)}{Bauxite}{2}{Bauxite_Mine}
		\prodnode{(2,-2.25)}{Aluminium_Profiles}{18}{Aluminium_Smelter}
		\connect{Coal}{Aluminium_Profiles}
		\connect{Bauxite}{Aluminium_Profiles}
		\prodnode{(2,0)}{Wood}{3}{Lumberjack\%27s_Hut_(New_World)}
		\consnode{(5,-1.125)}{Sewing_machines}{6}{Sewing_Machine_Factory}{Artisans/70/2/0/12,Engineers/35/6/0/32,Obreros/80/2/0/5,Artistas/80/5/0/16,Elders/111.11/2/0/6.4}			
		\connect{Wood}{Sewing_machines}
		\connect{Aluminium_Profiles}{Sewing_machines}
	\end{tikzpicture}}
	\tcbox[nodebox, %Jalea
		raster multicolumn=6, top=15mm,
		raster multirow=3]{
	\begin{tikzpicture}
		\prodnode{(0,-1.5)}{Calamari}{2}{Calamari_Fishery}
		\farmnode{(0,0)}{Corn}{2}{Corn_Farm}{1×1}{168}
		\farmnodec{(0,-3.75)}{Herbs}{1}{Herb_Garden}{1×1}{128}{Farmers/119/2/0/4.5,Elders/30.8/3/0/2.4}
		
		\consnode{(3.25,-1.5)}{Jalea}{1}{Jalea_Kitchen}{Artistas/60.6/2/0/12}
		\connect{Calamari}{Jalea}
		\connect{Corn}{Jalea}
		\connect{Herbs}{Jalea}
\end{tikzpicture}}
	\tcbox[nodebox, %Motors
		raster multicolumn=6,
		raster multirow=4]{
	\begin{tikzpicture}
		\prodnode{(0,0)}{Copper}{2}{Copper_Mine}
		\farmnode{(0,-1.5)}{Caoutchouc}{4}{Caoutchouc_Plantation}{1×1}{144}
		\prodnode{(2,-0.75)}{Electric_Cables}{2}{Cable_Factory}
		\connect{Copper}{Electric_Cables}
		\connect{Caoutchouc}{Electric_Cables}
		
		\prodnode{(0,-3)}{Iron}{1}{Iron_Mine}
		\prodnode{(0,-4.5)}{Charcoal_kiln}{2}{Charcoal_Kiln}
		\prodnode{(2,-3.75)}{Steel}{2}{Furnace}
		\connect{Charcoal_kiln}{Steel}
		\connect{Iron}{Steel}
		
		\chemnodenew{(2,-2.25)}{Celluloid}{2}
		\consnodefe{(5,-2.25)}{Motor}{2}{Motor_Assembly_Plant}{Technicians/133.3/5/2/0.8}
		\connect{Celluloid}{Motor}
		\connect{Electric_Cables}{Motor}
		\connect{Steel}{Motor}
	\end{tikzpicture}}
	\tcbox[nodebox, %Empty
		raster multicolumn=6,
		raster multirow=2]{}
	\tcbox[nodebox, %Fans
		raster multicolumn=6,
		raster multirow=2]{
	\begin{tikzpicture}
		\prodnodelbld{(0,-1.5)}{Charcoal_kiln}{3}{Charcoal_Kiln}{Coal}{\newworld}
		\prodnode{(0,-3)}{Bauxite}{1}{Bauxite_Mine}
		\prodnode{(2,-2.25)}{Aluminium_Profiles}{9}{Aluminium_Smelter}
		\connect{Coal}{Aluminium_Profiles}
		\connect{Bauxite}{Aluminium_Profiles}
		
		\prodnodefe{(2,0)}{Motor}{3}{Motor_Assembly_Plant}
		\consnodefe{(5,-1.125)}{Fans}{3}{Fan_Factory}{Artistas/641/3/0/32,Investors/133.33/10/0/10,Obreros/200/6/0/11.6,Scholars/200/5/0/2.4}				
		\connect{Motor}{Fans}
		\connect{Aluminium_Profiles}{Fans}
	\end{tikzpicture}}
	\tcbox[nodebox, %Scooters
		raster multicolumn=6, left=0mm,
		raster multirow=2]{
	\begin{tikzpicture}
		\farmnodep{(0,-0.75)}{Saltpeter}{4}{Alpaca_Farm}{3×3}{4}{Alpaca_wool}
		\prodnode{(0,-2.25)}{Minerals}{1}{Mineral_Mine}
		\prodnode{(2,-1.5)}{Pigments}{3}{Arsenal}
		\connect{Saltpeter}{Pigments}
		\connect{Minerals}{Pigments}
		
		\prodnodefe{(2,0)}{Motor}{2}{Motor_Assembly_Plant}
		\farmnode{(2,-3)}{Caoutchouc}{4}{Caoutchouc_Plantation}{1×1}{144}
		\consnodefe{(5,-1.5)}{Scooter}{4}{Scooter_Factory}{Artistas/575/5/0/36,Artisans/153.85/7/0/13.5,Scholars/80/7/0/3.6}				
		\connect{Motor}{Scooter}
		\connect{Caoutchouc}{Scooter}
		\connect{Pigments}{Scooter}
	\end{tikzpicture}}
\end{tcbraster}
	\end{tcolorbox}
\end{landscape}
\pagebreak

\begin{landscape} % SIDEWAYS PAGE
	\begin{tcolorbox}[raster height=\boxheight,
		height=\paperwidth-2cm, % subtract margins manually
		page box]
		\section{Public Services}
\begin{tcbraster}[raster columns=12, grid format]
	\tcbox[nodebox, %Fire_Extinguishers
		raster multicolumn=6, left=0mm,
		raster multirow=3]{
	\begin{tikzpicture}
		\prodnode{(0,0)}{Iron}{1}{Iron_Mine}
		\prodnode{(0,-1.5)}{Charcoal_kiln}{2}{Charcoal_Kiln}
		\prodnode{(2,-0.75)}{Steel}{2}{Furnace}
		\connect{Charcoal_kiln}{Steel}
		\connect{Iron}{Steel}
		\farmnode{(2,-2.25)}{Caoutchouc}{4}{Caoutchouc_Plantation}{1×1}{144}
		
		\prodnode{(4,-1.5)}{Fire_Extinguishers}{3}{Arsenal}
		\prodnode{(6,-1.5)}{Fire_Department}{24}{Fire_department}
		\connect{Caoutchouc}{Fire_Extinguishers}
		\connect{Steel}{Fire_Extinguishers}
		\connect{Fire_Extinguishers}{Fire_Department}
	\end{tikzpicture}}
	\tcbox[nodebox, %Film Reals
	raster multicolumn=6, left=-4mm,
	raster multirow=3]{
	\begin{tikzpicture}
		\chemnodenew{(0,-2)}{Celluloid}{2}
		\farmnodep{(0,0)}{Saltpeter}{4}{Alpaca_Farm}{3×3}{4}{Alpaca_wool}
		\chemnodenewcons{(2.5,-1)}{film_reel}{8}{Investors/89/5/0/8.5,Scholars/80/4/0/4.8}
		\consnode{(5.85,-1)}{Cinema}{12}{Cinema}{Artistas/-/0/4/12,Jornaleros/-/2/0/1.6}
		\connect{Celluloid}{film_reel}
		\connect{Saltpeter}{film_reel}
		\connect{film_reel}{Cinema}
	\end{tikzpicture}}
	\tcbox[nodebox, %Police_Equipment
		raster multicolumn=6, left=0mm,
		raster multirow=3]{
	\begin{tikzpicture}
		\prodnode{(2,-2.25)}{Wood}{1}{Lumberjack\%27s_Hut_(Old_World)}
		\prodnode{(0,0)}{Iron}{1}{Iron_Mine}
		\prodnode{(0,-1.5)}{Charcoal_kiln}{2}{Charcoal_Kiln}
		\prodnode{(2,-0.75)}{Steel}{2}{Furnace}
		\connect{Charcoal_kiln}{Steel}
		\connect{Iron}{Steel}
		\farmnode{(0,-3.75)}{Cotton}{2}{Cotton_Plantation}{1×1}{144}
		\prodnode{(2,-3.75)}{Cotton_fabric}{1}{Cotton_Mill}
		\connect{Cotton}{Cotton_fabric}
		
		\prodnode{(4,-2.25)}{Police_Equipment}{4}{Supply_Factory}
		\prodnode{(6,-2.25)}{Police_Headquarters}{24}{Police_Headquarters}
		\connect{Steel}{Police_Equipment}
		\connect{Wood}{Police_Equipment}
		\connect{Cotton_fabric}{Police_Equipment}
		\connect{Police_Equipment}{Police_Headquarters}
	\end{tikzpicture}}
	\tcbox[nodebox, %Costumes
		raster multicolumn=6, left=-4mm, top=0mm,
		raster multirow=3]{
	\begin{tikzpicture}
		\farmnode{(0,0)}{Cotton}{4}{Cotton_Plantation}{1×1}{144}
		\prodnode{(2,0)}{Cotton_fabric}{2}{Cotton_Mill}
		\connect{Cotton}{Cotton_fabric}
		\farmnodep{(2,-1.5)}{Nandu_Feathers}{2}{Nandu_Farm}{4×4}{4}{nandu_leather}
		\farmnodep{(0,-3.75)}{Saltpeter}{4}{Alpaca_Farm}{3×3}{4}{Alpaca_wool}
		\prodnode{(0,-2.25)}{Minerals}{1}{Mineral_Mine}
		\prodnode{(2,-3)}{Pigments}{2}{Arsenal}
		\connect{Saltpeter}{Pigments}
		\connect{Minerals}{Pigments}
		
		\prodnode{(4,-1.5)}{Costumes}{4}{Costume_Shop}
		\consnode{(6,-1.5)}{samba_school}{8}{Samba_School}{Artistas/-/0/4/12,Obreros/-/2/0/2.8}
		\connect{Cotton_fabric}{Costumes}
		\connect{Nandu_Feathers}{Costumes}
		\connect{Pigments}{Costumes}
		\connect{Costumes}{samba_school}
	\end{tikzpicture}}
	\tcbox[nodebox, %Medicine
		raster multicolumn=6, left=0mm,
		raster multirow=3]{
	\begin{tikzpicture}
		\prodnode{(0,-0.75)}{Wood}{1}{Lumberjack\%27s_Hut_(Old_World)}
		\farmnode{(0,-2.25)}{Corn}{4}{Corn_Farm}{1×1}{168}
		\chemnodenew{(2,-1.5)}{Ethanol}{2}
		\connect{Wood}{Ethanol}
		\connect{Corn}{Ethanol}
		\farmnode{(2,0)}{Herbs}{1}{Herb_Garden}{1×1}{128}
		\farmnode{(2,-3.25)}{Orchid}{2}{Orchid_Farm}{1×1}{128}
		
		\consnode{(4.75,-1.5)}{Medicine}{3}{Arsenal}{engineers/80/7/0/6.4}
		\prodnode{(6.75,-1.5)}{City_Hospital}{24}{City_Hospital}
		\connect{Orchid}{Medicine}
		\connect{Ethanol}{Medicine}
		\connect{Herbs}{Medicine}
		\connect{Medicine}{City_Hospital}
	\end{tikzpicture}}
\end{tcbraster}
		\tcbline
		\include{dropSupplies.tex}
	\end{tcolorbox}
\end{landscape}
\pagebreak



\begin{landscape} % SIDEWAYS PAGE
	\begin{tcolorbox}[raster height=\boxheight,
	height=\paperwidth-2cm, % subtract margins manually
	page box]
		\section{Tourist Consumables}
\begin{tcbraster}[raster columns=\portraitraster,
	grid format]
	\tcbox[nodebox, %Jam
		raster multicolumn=1,
		raster multirow=3]{
	\begin{tikzpicture}
		\orchnodeoldcons{(0,0)}{jam}{1}{Tourists/1000}
	\end{tikzpicture}}
	\tcbox[nodebox, %Lemonade
		raster multicolumn=5,
		raster multirow=3]{
	\begin{tikzpicture}
		\orchnodenew{(4,0)}{Coconut_Oil}{1}
		\orchnodenew{(4,-1.5)}{Cinnamon}{1}
		\farmnode{(0,-3)}{Pigs}{2}{Pig_Farm}{2×3}{5}
		\prodnode{(2,-3)}{Tallow}{2}{Rendering_Works}
		\prodnode{(4,-3)}{Soap}{1}{Soap_Factory}
		\connect{Pigs}{Tallow}
		\connect{Tallow}{Soap}
		\chemnodeoldcons{(6,-1.5)}{Shampoo}{1}{Tourists/2000}
		\connect{Coconut_Oil}{Shampoo}
		\connect{Cinnamon}{Shampoo}
		\connect{Soap}{Shampoo}
	\end{tikzpicture}}
	\tcbox[nodebox, %Lemonade
		raster multicolumn=3,
		raster multirow=3]{
	\begin{tikzpicture}
        \farmnode{(0,0)}{Sugar_cane}{1}{Sugar_Cane_Plantation}{1×1}{128}
		\prodnode{(2,0)}{Sugar}{1}{Sugar_Refinery}
		\connect{Sugar_cane}{Sugar}
		\orchnodenew{(2,-1.5)}{Citrus}{1}
        \prodnode{(2,-3)}{Saltpeter}{4}{Saltpeter_Works}
		\chemnodeoldcons{(4,-1.5)}{Lemonade}{1}{Tourists/1000}
		\connect{Sugar}{Lemonade}
		\connect{Citrus}{Lemonade}
		\connect{Saltpeter}{Lemonade}
	\end{tikzpicture}}
	\tcbox[nodebox, %Souvenirs
		raster multicolumn=3,
		raster multirow=3]{
	\begin{tikzpicture}
		\orchnodenew{(2,-1.5)}{Camphor_Wax}{1}
		\prodnode{(0,0)}{Quartz_sand}{1}{Sand_Mine}
		\prodnode{(2,0)}{Glass}{1}{Glassmakers}
		\connect{Quartz_sand}{Glass}
		\farmnode{(2,-3)}{Cotton}{2}{Cotton_Plantation}{1×1}{144}
		\chemnodeoldcons{(4,-1.5)}{Souvenirs}{1}{Tourists/500}
		\connect{Camphor_Wax}{Souvenirs}
		\connect{Glass}{Souvenirs}
		\connect{Cotton}{Souvenirs}
	\end{tikzpicture}}
\end{tcbraster}

\tcbline


		\tcbline
		\section{Skyscrapers Materials}
\begin{tcbraster}[raster columns=\portraitraster,
	grid format]
		\tcbox[nodebox, %Ethanol
		raster multicolumn=3,
		raster multirow=2]{
			\begin{tikzpicture}
				\prodnode{(0,-2.5)}{Wood}{1}{Lumberjack\%27s_Hut_(Old_World)}
				\chemnodenewc{(2,-3.5)}{Ethanol}{2}{15}
				\farmnode{(0,-4.5)}{Corn}{4}{Corn_Farm}{1×1}{168}
				\connect{Wood}{Ethanol}
				\connect{Corn}{Ethanol}
		\end{tikzpicture}}	
		\tcbox[nodebox, %Celluloid
		raster multicolumn=3,
		raster multirow=2]{
			\begin{tikzpicture}
				\farmnode{(0,0)}{Cotton}{2}{Cotton_Plantation}{1×1}{144}
				\orchnodenew{(0,-1.5)}{Camphor_wax}{1}
				\chemnodenew{(0,-3)}{Ethanol}{1}
				\chemnodenewc{(2,-1.5)}{Celluloid}{1}{30}
				\connect{Cotton}{Celluloid}
				\connect{Camphor_wax}{Celluloid}
				\connect{Ethanol}{Celluloid}
		\end{tikzpicture}}	
	\tcbox[nodebox, %Elavators
		raster multicolumn=3,
		raster multirow=4]{
	\begin{tikzpicture}
		\prodnode{(0.75,0)}{Iron}{1}{Iron_Mine}
		\prodnode{(0,-1.5)}{Charcoal_kiln}{2}{Charcoal_Kiln}
		\prodnode{(1.5,-1.5)}{Coal}{1}{Coal_Mine}
		\divider{Charcoal_kiln}{Coal}
		\prodnode{(3,-0.75)}{Steel}{2}{Furnace}
		\connect{Iron}{Steel}
		\connect{Coal}{Steel}
		
		\prodnode{(0.75,-3.5)}{Wood}{1}{Lumberjack\%27s_Hut_(Old_World)}
		\prodnode{(3,-3.5)}{Wood_veneers}{4}{Marquetry_Workshop_(Old_World)}
		\connect{Wood}{Wood_veneers}
		
		\prodnodefe{(3, -5.5)}{Steam_motors}{3}{Motor_Assembly_Line}
		\asslnodec{(5,-3.5)}{Elevators}{1}{15}
		\connect{Steel}{Elevators}
		\connect{Wood_veneers}{Elevators}
		\connect{Steam_motors}{Elevators}
	\end{tikzpicture}}	
	\tcbox[nodebox, %Lacquer
		raster multicolumn=3,
		raster multirow=3]{
	\begin{tikzpicture}
		\chemnodenew{(0,-3)}{Ethanol}{1}
		\orchnodeold{(0,-1.5)}{Resin}{1}
		\prodnode{(0,-0)}{Quartz_sand}{1}{Sand_Mine}
		\chemnodeoldc{(2,-1.5)}{Lacquer}{1}{30}
		\connect{Ethanol}{Lacquer}
		\connect{Resin}{Lacquer}
		\connect{Quartz_sand}{Lacquer}
\end{tikzpicture}}		
\end{tcbraster}
		\tcbline
	\end{tcolorbox}
\end{landscape}
\pagebreak

\begin{landscape} % SIDEWAYS PAGE
	\begin{tcolorbox}[raster height=\boxheight,
	height=\paperwidth-2cm, % subtract margins manually
	page box]
		\section{Engineer Skyscrapers Consumables}
\begin{tcbraster}[raster columns=\portraitraster,
	grid format]
	
	\tcbox[nodebox, %Type_Writers
		raster multicolumn=3, left=-3mm,
		raster multirow=5]{
	\begin{tikzpicture}
		\prodnode{(0.75,0)}{Iron}{1}{Iron_Mine}
		\prodnode{(0,-1.5)}{Charcoal_kiln}{2}{Charcoal_Kiln}
		\prodnode{(1.5,-1.5)}{Coal}{1}{Coal_Mine}
		\divider{Charcoal_kiln}{Coal}
		\prodnode{(3,-0.75)}{Steel}{2}{Furnace}
		\connect{Iron}{Steel}
		\connect{Coal}{Steel}
		
	
		\prodnode{(0.75,-3)}{Zinc}{2}{Zinc_Mine}
		\prodnode{(0.75,-4.5)}{Copper}{2}{Copper_Mine}
		\prodnode{(2.5, -3.75)}{Brass}{4}{Brass_Smeltery}
		\connect{Zinc}{Brass}
		\connect{Copper}{Brass}
		
		\chemnodeold{(2.5,-6)}{Lacquer}{2}
		\asslnodecons{(5.5,-3.75)}{Typewriters}{1}{Engineers/500/10/0/38,Investors/285.6/10/0/38,Obreros/500/7/0/7}
		\connect{Brass}{Typewriters}
		\connect{Steel}{Typewriters}
		\connect{Lacquer}{Typewriters}
	\end{tikzpicture}}
	\tcbox[nodebox, %Violins
		raster multicolumn=3,left=-3mm,
		raster multirow=5]{
	\begin{tikzpicture}
		\prodnode{(0.75,0)}{Iron}{1}{Iron_Mine}
		\prodnode{(0,-1.5)}{Charcoal_kiln}{2}{Charcoal_Kiln}
		\prodnode{(1.5,-1.5)}{Coal}{1}{Coal_Mine}
		\divider{Charcoal_kiln}{Coal}
		\prodnode{(3,-0.75)}{Steel}{2}{Furnace}
		\connect{Iron}{Steel}
		\connect{Coal}{Steel}
		
		\orchnodenew{(2.5,-3.5)}{Cherry_Wood}{2}
		
		\chemnodeold{(2.5,-6)}{Lacquer}{2}
		\arwsnodecons{(5.5,-3.75)}{Violins}{4}{Engineers/333.3/5/0/28.8,Investors/314.4/5/0/21}
		\connect{Cherry_Wood}{Violins}
		\connect{Steel}{Violins}
		\connect{Lacquer}{Violins}
	\end{tikzpicture}}	
	\tcbox[nodebox, %Chewing_Gum
		raster multicolumn=3,
		raster multirow=3]{
	\begin{tikzpicture}
		\farmnode{(0,0)}{Sugar_cane}{1}{Sugar_Cane_Plantation}{1×1}{128}
		\prodnode{(2,0)}{Sugar}{1}{Sugar_Refinery}
		\connect{Sugar_cane}{Sugar}
		\orchnodenew{(2,-1.5)}{Cinnamon}{1}
		\farmnode{(2,-3)}{Caoutchouc}{2}{Caoutchouc_Plantation}{1×1}{144}
		\chemnodenewcons{(5,-1.5)}{Chewing_Gum}{1}{Engineers/384.6/10/0/17.2,Investors/363.6/10/0/12.5}
		\connect{Sugar}{Chewing_Gum}
		\connect{Cinnamon}{Chewing_Gum}
		\connect{Caoutchouc}{Chewing_Gum}
	\end{tikzpicture}}	
\end{tcbraster}
		\tcbline
		\include{skyinvestors.tex}
	\end{tcolorbox}
\end{landscape}
\pagebreak

\begin{tcolorbox}[height fill,
	raster height=\boxheight,
	page box]
	\include{recipes.tex}
\end{tcolorbox}
\pagebreak

\begin{tcolorbox}[height fill,
	raster height=\boxheight,
	page box]
	\include{patents.tex}
\end{tcolorbox}
\pagebreak

\begin{tcolorbox}[height fill,
	raster height=\boxheight,
	page box]
	\include{ironTower.tex}
\end{tcolorbox}
\pagebreak

\begin{landscape} % SIDEWAYS PAGE
	\begin{tcolorbox}[height fill,
		raster height=\boxheight,
		page box]
		\include{palace.tex}
		\tcbline
	\end{tcolorbox}
\end{landscape} % SIDEWAYS PAGE
\pagebreak

\begin{tcolorbox}[height fill,
	raster height=\boxheight,
	page box]
	\section{Zoo}
\begin{tcbraster}[raster columns=5, 
	grid format]
	\tcbox[nodebox, %skip
 		raster multicolumn=1,
	 	raster multirow=3]{
	\begin{tikzpicture}
		\consnode{(0,0)}{zoo}{1}{Zoo}{Tourists/-/0/3/375}
	\end{tikzpicture} 	
 	
 	}
	\tcbox[blankest,
		raster multicolumn=4,left=0mm, top=-5mm,
		raster multirow=3]{ 
		\begin{tcbraster}[raster height=\boxheight,
			raster columns=4,
			grid format]
		\tcbox[nodebox, %Luminaries
		raster multicolumn=1, top=-2mm,
		raster multirow=2]{
			\begin{tikzpicture}
				\cultnode{(0,0)}{Dragonfish}{30}
				\cultnode{(1.5,0)}{Firefly_Squid}{30}
				\cultnodelbld{(3,0)}{Luminescent_Fish}{30}{2205}
			\end{tikzpicture}
			+3\smallicon{Attractiveness} from \smallicon{Pub},\smallicon{Variety_Theatre},\smallicon{Members_Club},\smallicon{department_store},\smallicon{drug_store},\smallicon{furniture_store},\smallicon{Restaurant}
		}	
		\tcbox[nodebox, %Arctic Tundra
		raster multicolumn=1,
		raster multirow=2]{
			\begin{tikzpicture}
				\cultnode{(0,0)}{arctic_wolf}{30}
				\cultnode{(1.5,0)}{musk_oxen}{30}
				\cultnodelbld{(3,0)}{reindeer}{30}{2205}
			\end{tikzpicture}
			 Public Moorings:\\ 
			 Increased Visits: +5\%\\
 			 +100\smallicon{Attractiveness}\\
		}
		\tcbox[nodebox, %Teeming Lakes
		raster multicolumn=1,
		raster multirow=2]{
			\begin{tikzpicture}
				\cultnode{(0,0)}{flamingo}{20}
				\cultnode{(1.5,0)}{grey_crowned_crane}{30}
				\cultnode{(3,0)}{hippotanamus}{40}
			\end{tikzpicture}
			all Coastal Buildings:\\
			+20\%\smallicon{Productivity}
		}	
		\tcbox[nodebox, %Great Desert
		raster multicolumn=1,
		raster multirow=2]{
			\begin{tikzpicture}
				\cultnode{(0,0)}{dromedary}{20}
				\cultnode{(1.5,0)}{fennec}{30}
				\cultnode{(3,0)}{scorpion}{30}
			\end{tikzpicture}
			Fire Station:\\
			Extinguish Speed: +30\%\\
			Movement Speed: +10\%
		}		
		\tcbox[nodebox, %Cordillera
		raster multicolumn=1, top=-2mm,
		raster multirow=2]{
			\begin{tikzpicture}
				\cultnode{(0,0)}{alpaca}{20}
				\cultnode{(1.5,0)}{spectacled_bear}{40}
				\cultnodelbld{(3,0)}{vulture}{20}{1701}
			\end{tikzpicture}
			Farmers and Jornaleros:\\
			+10\%\smallicon{workforce}	+10\%\smallicon{credits}
		}	
		\tcbox[nodebox, %Taiga Forest
		raster multicolumn=1,
		raster multirow=2]{
			\begin{tikzpicture}
				\cultnode{(0,0)}{black_bear}{20}
				\cultnode{(1.5,0)}{brown_bear}{20}
				\cultnodelbld{(3,0)}{elk}{30}{2205}
			\end{tikzpicture}
			 Residents consume 5\% less \smallicon{Fish},\smallicon{Bread},\smallicon{Canned_Food},\smallicon{Sausages},\smallicon{Chocolate},\smallicon{Fried_Plantains},\smallicon{Tortilla},\smallicon{Seafood_Stew}.
		}	
		\tcbox[nodebox, %Extinct Species
		raster multicolumn=1, top=0mm,
		raster multirow=2]{
			\begin{tikzpicture}
				\cultnode{(0,0)}{dire_wolf}{50}
				\cultnode{(1.5,0)}{giant_deer}{50}
				\cultnodelbld{(3,0)}{kolponomos}{20}{1800}
			\end{tikzpicture}
			Zoological Gardens:\\
			+10\%\smallicon{Attractiveness}
		}		
		\tcbox[nodebox, %Ocean Predators
		raster multicolumn=1,
		raster multirow=2]{
			\begin{tikzpicture}
				\cultnode{(0,0)}{orca}{30}
				\cultnode{(1.5,0)}{shark}{40}
				\cultnodelbld{(3,0)}{swordfish}{30}{2070}
			\end{tikzpicture}
			All Coastal Buildings:\\
			+10\%\smallicon{productivity} -5\%\smallicon{workforce} -10\%\smallicon{balance}
		}		
		\end{tcbraster}
	}
	\tcbox[nodebox, %Eastern Jungle
	raster multicolumn=1,top=-4mm,
	raster multirow=3]{
		\begin{tikzpicture}
			\cultnode{(0,0)}{chital}{20}
			\cultnode{(1.5,0)}{crocodile}{20}
			\cultnodelbld{(3,0)}{eastern_elephant}{40}{1404}
			\cultnode{(0,-1.5)}{waterbuffalo}{20}
			\cultnode{(1.5,-1.5)}{tiger}{30}
			\cultnode{(3,-1.5)}{peacock}{20}
		\end{tikzpicture}
		Harbor Activity +100\%. chance of gaining 5t\\
		 \smallicon{Coffee_Beans},\smallicon{Sugar_Cane},\smallicon{Tobacco},\smallicon{Cotton},\smallicon{Cocoa}, \smallicon{Plantains},\smallicon{Pearls},\smallicon{Chocolate},\smallicon{Coffee},\smallicon{Fried_Plantains}, \smallicon{Tortilla},\smallicon{Sugar}.\\
	}
	\tcbox[nodebox, % Rainforest
	raster multicolumn=1,top=12mm,
	raster multirow=2]{
		\begin{tikzpicture}
			\cultnode{(0,0)}{amazonas_dolphin}{40}
			\cultnode{(1.5,0)}{black_panther}{40}
			\cultnodelbld{(3,0)}{black_caiman}{30}{1701}
			\cultnode{(0,-1.5)}{jaguar}{30}
			\cultnode{(1.5,-1.5)}{pelican}{30}
			\cultnode{(3,-1.5)}{puerto_rican_amazon}{40}
		\end{tikzpicture}
		all Trade Unions:\\
		+30\smallicon{Attractiveness} 
	}	
	\tcbox[nodebox, %Snowflakes
	raster multicolumn=1,top=12mm,
	raster multirow=2]{
		\begin{tikzpicture}
			\cultnode{(0,0)}{albinos_fennec}{50}
			\cultnode{(1.5,0)}{albino_crocodile}{40}
			\cultnodelbld{(3,0)}{albino_gorilla}{50}{2205}
			\cultnode{(0,-1.5)}{albino_peacock}{50}
			\cultnode{(1.5,-1.5)}{snow_tiger}{50}
			\cultnode{(3,-1.5)}{white_flag_dolphin}{50}
		\end{tikzpicture}
		Zoological Gardens:\\
		-50\% \smallicon{balance} +20\%\smallicon{Attractiveness} 
	}		
	\tcbox[nodebox, %Domestic Animals
	raster multicolumn=1,top=12mm,
	raster multirow=2]{
		\begin{tikzpicture}
			\cultnode{(0,0)}{cattle}{10}
			\cultnode{(1.5,0)}{chicken}{10}
			\cultnode{(3,0)}{goat}{10}
			\cultnode{(0,-1.5)}{horse}{10}
			\cultnode{(1.5,-1.5)}{domestic_pig}{10}
			\cultnode{(3,-1.5)}{sheep}{10}
		\end{tikzpicture}
		All Animal Farms:\\
		+5\%\smallicon{Productivity} -10\% \smallicon{balance}.
	}
	\tcbox[nodebox, %Great Coral Reef
	raster multicolumn=1,top=12mm,
	raster multirow=2]{
		\begin{tikzpicture}
			\cultnode{(0,0)}{balloonfish}{30}
			\cultnode{(1.5,0)}{dolphin}{30}
			\cultnodelbld{(3,0)}{turtle}{10}{1404}
			\cultnode{(0,-1.5)}{lionfish}{20}
			\cultnode{(1.5,-1.5)}{mantaray}{20}
			\cultnode{(3,-1.5)}{seahorses}{20}
		\end{tikzpicture}
		Workers, Artisans, Obreros:\\
		+2\smallicon{Happiness_positive} +2\smallicon{Attractiveness}
	}
	\tcbox[nodebox, %Abyssal Depths
	raster multicolumn=1, top=5mm,
	raster multirow=3]{
		\begin{tikzpicture}
			\cultnode{(0,0)}{anglerfish}{40}
			\cultnode{(1.5,0)}{blackswallower}{40}
			\cultnodelbld{(3,0)}{blob_fish}{30}{2070}
			\cultnode{(0,-1.5)}{goblinshark}{40}
			\cultnode{(1.5,-1.5)}{helmet_jellyfish}{30}
			\cultnode{(3,-1.5)}{kraken}{50}
		\end{tikzpicture}
		Salvager get increased chance of finding Epic and Legendary animals.
	}		
	\tcbox[nodebox, %Polar Circle
	raster multicolumn=1, top=2mm,
	raster multirow=3]{
		\begin{tikzpicture}
			\cultnode{(0,0)}{arctic_fox}{20}
			\cultnode{(1.5,0)}{great_auk}{50}
			\cultnodelbld{(3,0)}{narwhal}{50}{2205}
			\cultnode{(0,-1.5)}{polar_bear}{40}
			\cultnode{(1.5,-1.5)}{sealion}{20}
			\cultnode{(3,-1.5)}{walrus}{40}
		\end{tikzpicture}
		All Cloth Industries:\\
	    +25\%\smallicon{Productivity} -25\%\smallicon{balance}.
	}
	\tcbox[nodebox, %Enbesan Highlands
	raster multicolumn=1,top=5mm,
	raster multirow=3]{
		\begin{tikzpicture}
			\cultnode{(0,0)}{two_horned_chameleon}{20}
			\cultnode{(1.5,0)}{oryx}{30}
			\cultnode{(3,0)}{mountain_nyala}{30}
			\cultnode{(0,-1.5)}{gelada}{40}
			\cultnode{(1.5,-1.5)}{bearded_vulture}{40}
			\cultnode{(3,-1.5)}{enbesan_wolf}{20}
		\end{tikzpicture}
		Scholars:\\
		-10\%\smallicon{Seafood_Stew},\smallicon{Hibiscus_Tea}
	}		
	\tcbox[nodebox, %Proud Savannah
	raster multicolumn=1,top=5mm,
	raster multirow=3]{
		\begin{tikzpicture}
			\cultnode{(0,0)}{cheeta}{40}
			\cultnode{(1.5,0)}{lion}{50}
			\cultnode{(3,0)}{ostrich}{20}
			\cultnode{(0,-1.5)}{spotted_hyena}{30}
			\cultnode{(1.5,-1.5)}{wildebeest}{30}
			\cultnode{(3,-1.5)}{zebra}{20}
		\end{tikzpicture}
		Cattle Farm, Alpaca Farm:\\
		No. of Modules: -25\%
	}	
	\tcbox[nodebox, % Miombo Woodlands
	raster multicolumn=1, top=5mm,
	raster multirow=3]{
		\begin{tikzpicture}
			\cultnode{(0,0)}{black_rhino}{40}
			\cultnode{(1.5,0)}{caracal}{40}
			\cultnode{(3,0)}{elephant}{30}
			\cultnode{(0,-1.5)}{giraffe}{40}
			\cultnode{(1.5,-1.5)}{impala}{30}
			\cultnode{(3,-1.5)}{wild_dog}{50}
		\end{tikzpicture}
		Lumberjack's Hut, Charcoal Kiln,  Hunting Cabin:\\
		-10\%\smallicon{Forest_Density} +5\smallicon{Attractiveness}
	}	
\end{tcbraster}


	\tcbline
\end{tcolorbox}
\pagebreak

\begin{tcolorbox}[height fill,
	raster height=\boxheight,
	page box]
	\section{Museum}
\begin{tcbraster}[raster columns=5, 
	grid format]
	\tcbox[nodebox, %skip
 		raster multicolumn=1,
	 	raster multirow=3]{
		\begin{tikzpicture}
		\consnode{(0,0)}{museum}{1}{museum}{Tourists/-/0/3/375}
		\end{tikzpicture}  	
 	}
	\tcbox[blankest,
		raster multicolumn=4,left=0mm, top=-15mm,
		raster multirow=3]{ 
		\begin{tcbraster}[raster height=\boxheight,
			raster columns=4,
			grid format]	
		\tcbox[nodebox, %Battle of Trelawney
		raster multicolumn=1,
		raster multirow=2]{
			\begin{tikzpicture}
				\cultnode{(0,0)}{empire_figurehead}{40}
				\cultnode{(1.5,0)}{nadaskys_medal}{40}
				\cultnode{(3,0)}{nadaskys_sword}{40}
			\end{tikzpicture}
			 +20\% increased damage to sailing ships
		}
		\tcbox[nodebox, %Empire of the Eagle
		raster multicolumn=1,
		raster multirow=2]{
			\begin{tikzpicture}
				\cultnode{(0,0)}{eagle_mosaik}{40}
				\cultnode{(1.5,0)}{roman_sanctuary}{40}
				\cultnodelbld{(3,0)}{augustan_forum}{50}{1800}
			\end{tikzpicture}
			All Residences:\\
			+5\%\smallicon{credits}\\
			-10\%\smallicon{Chance_of_Fire},\smallicon{Chance_of_Illness},\smallicon{Chance_of_Riots},\smallicon{Chance_of_Explosion} 
	    }	
		\tcbox[nodebox, %Gods of the Delta
		raster multicolumn=1,
		raster multirow=2]{
			\begin{tikzpicture}
				\cultnode{(0,0)}{anpu_statue}{40}
				\cultnode{(1.5,0)}{hor_statue}{40}
				\cultnodelbld{(3,0)}{temple_of_nafir}{50}{1800}
			\end{tikzpicture}
			Town Halls \& Trade Unions:\\
			+30\smallicon{Attractiveness} +10\smallicon{Influence}  
		}		
		\tcbox[nodebox, %Heirlooms of the Gold-Realm
		raster multicolumn=1,
		raster multirow=2]{
			\begin{tikzpicture}
				\cultnode{(0,0)}{mansa_musas_souvenir}{30}
				\cultnode{(1.5,0)}{mappamundi}{30}
				\cultnode{(3,0)}{model_of_tambacto}{50}
			\end{tikzpicture}
			Harbor Activity: +20\%.\\ 
			Gain 5t of \smallicon{Gold},\smallicon{Jewelry}
		}		
		\tcbox[nodebox, %Icebound
			raster multicolumn=1,
			raster multirow=2]{
			\begin{tikzpicture}
				\cultnode{(0,0)}{mammoth}{50}
				\cultnode{(1.5,0)}{wolf_mummy}{40}
				\cultnodelbld{(3,0)}{expedition_relic}{30}{1800}
			\end{tikzpicture}
		    All Food\&Drinks Production:\\
		    -20\%\smallicon{credits}
		}
		\tcbox[nodebox, %New World Huaca
		raster multicolumn=1,
		raster multirow=2]{
			\begin{tikzpicture}
				\cultnode{(0,0)}{monkey_amulet}{20}
				\cultnode{(1.5,0)}{mummy_nazcan}{20}
				\cultnodelbld{(3,0)}{nascan_tomb}{30}{1701}
			\end{tikzpicture}
			 Public Moorings:\\ 
			 Increased Visits: +5\%
			 \hyperlink{https://anno1800.fandom.com/wiki/New_World_Huaca_specialists_pool}{unique item pool} 
		}		
		\tcbox[nodebox, %Skull and Bones
		raster multicolumn=1,
		raster multirow=2]{
			\begin{tikzpicture}
				\cultnode{(0,0)}{treasure_ben_sahid}{40}
				\cultnode{(1.5,0)}{redbeard_skeleton}{40}
				\cultnode{(3,0)}{pirate_sword}{40}
			\end{tikzpicture}
			All warships:\\
			+3\% increased speed\\
			-5\% maintenance cost\\
			-10\% damage slowdown 
		}	
		\tcbox[nodebox, %Origin of Mankind
		raster multicolumn=1,
		raster multirow=2]{
			\begin{tikzpicture}
				\cultnode{(0,0)}{lascaux_painting}{40}
				\cultnode{(1.5,0)}{neanderthaler}{50}
				\cultnodelbld{(3,0)}{Valojoulx_painting}{50}{1800}
			\end{tikzpicture}
			Fisheries, Crop/Animal Farms, Hunting Cabins:\\
			+10\%\smallicon{productivity} -5\%\smallicon{workforce} -5\%\smallicon{balance}
		}	
		\end{tcbraster}
	}
	\tcbox[nodebox, % Thule Relics
	raster multicolumn=1,top=-4mm,
	raster multirow=3]{
		\begin{tikzpicture}
			\cultnode{(0,0)}{flying_bear}{20}
			\cultnode{(1.5,0)}{igluit}{40}
			\cultnodelbld{(3,0)}{inukshuk}{30}{2070}
			\cultnode{(0,-1.5)}{pirujaqarvik}{30}
			\cultnode{(1.5,-1.5)}{qamutiik_toy}{30}
			\cultnode{(3,-1.5)}{qilakitsoq_mummy}{40}
		\end{tikzpicture}
		Taiga forest, polar circle, arctic tundra, northern sagas, bronze age, thule relics, icebound, and sub-alpine:\\
		+20\%\smallicon{Attractiveness}
	}	
	\tcbox[nodebox, % Lost Tribes
	raster multicolumn=1, top=12mm,
	raster multirow=3]{
		\begin{tikzpicture}
			\cultnode{(0,0)}{bovine_ivory_ring}{10}
			\cultnode{(1.5,0)}{chiwara_figure}{20}
			\cultnode{(3,0)}{giwoyo_mask}{20}
			\cultnode{(0,-1.5)}{nuk_terracota_figure}{30}
			\cultnode{(1.5,-1.5)}{phemba}{30}
			\cultnode{(3,-1.5)}{yarobu_carving}{30}
		\end{tikzpicture}
		All Residences:\\
		-5\%\smallicon{Sewing_Machines},\smallicon{Canned_Food},\smallicon{Coffee}.
	}	
	\tcbox[nodebox, %Roots of Enbesa
	raster multicolumn=1,top=12mm,
	raster multirow=3]{
		\begin{tikzpicture}
			\cultnode{(0,0)}{ark_of_the_covenant}{50}
			\cultnode{(1.5,0)}{church_kidusi_antoni}{40}
			\cultnode{(3,0)}{fasil_ghebbi}{40}
			\cultnode{(0,-1.5)}{kebra_nagast}{20}
			\cultnode{(1.5,-1.5)}{lion_of_judah}{30}
			\cultnode{(3,-1.5)}{obelix_of_arksam}{30}
		\end{tikzpicture}
		Scholar Residences:\\
		 +1\smallicon{research}
	}
	\tcbox[nodebox, %Jurassic
	raster multicolumn=1,top=12mm,
	raster multirow=3]{
		\begin{tikzpicture}
			\cultnode{(0,0)}{dinosaur_egg}{40}
			\cultnode{(1.5,0)}{fossils}{30}
			\cultnodelbld{(3,0)}{mosasaurus}{50}{1800}
			\cultnode{(0,-1.5)}{sauropodon}{50}
			\cultnode{(1.5,-1.5)}{stegosaurus}{40}
			\cultnode{(3,-1.5)}{t_rex}{50}
		\end{tikzpicture}
		All Residences:\\
		+10\%\smallicon{credits} +10\smallicon{Happiness_positive} \\
	}	
	\tcbox[nodebox, %Bronze Age
	raster multicolumn=1,top=22mm,
	raster multirow=3]{
		\begin{tikzpicture}
			\cultnode{(0,0)}{celtic_carving}{40}
			\cultnode{(1.5,0)}{celtic_dolmen}{40}
			\cultnodelbld{(3,0)}{celtic_menhirs}{40}{2070}
			\cultnode{(0,-1.5)}{celts_dyrnwyn}{50}
			\cultnode{(1.5,-1.5)}{celts_stonehenge}{50}
			\cultnode{(3,-1.5)}{tara_brooch}{30}
		\end{tikzpicture}
		Lumberjacks' Huts, Hunting Cabins, Quarries and Orchards:\\
		+5\%\smallicon{productivity} +5\smallicon{Attractiveness}\\
		extra goods (1/10)
	}
	\tcbox[nodebox, %Lost Cities
	raster multicolumn=1,top=5mm,
	raster multirow=3]{
		\begin{tikzpicture}
			\cultnode{(0,0)}{mayan_calendar}{50}
			\cultnode{(1.5,0)}{mayan_glyphs}{40}
			\cultnodelbld{(3,0)}{mayan_house}{40}{1701}
			\cultnode{(0,-1.5)}{mayan_seaport}{50}
			\cultnode{(1.5,-1.5)}{mayan_temple}{50}
			\cultnode{(3,-1.5)}{takal_necklace}{40}
		\end{tikzpicture}
		Harbour Activity +100\%.\\
		gain 5t of \smallicon{Gold_Ore},\smallicon{Gold},\smallicon{Jewelry}
	}	
	\tcbox[nodebox, %Aegean Cultures
		raster multicolumn=1,top=5mm,
		raster multirow=3]{
	\begin{tikzpicture}
		\cultnode{(0,0)}{corinthian_mosaik}{40}
		\cultnode{(1.5,0)}{corinthian_olympean}{40}
		\cultnodelbld{(3,0)}{corinthian_temple_of_posseidon}{40}{1800}
		\cultnode{(0,-1.5)}{scroll_anthology_myths}{40}
		\cultnode{(1.5,-1.5)}{scroll_odyssey}{40}
		\cultnode{(3,-1.5)}{statue_marble_statue}{30}
	\end{tikzpicture}
	Variety Theatres, Museums and Radio Towers:\\
	+5\smallicon{Attractiveness} -20\%\smallicon{balance}. 
	}	
	\tcbox[nodebox, %Atlantean
		raster multicolumn=1, top=5mm,
		raster multirow=3]{
	\begin{tikzpicture}
		\cultnode{(0,0)}{atlantis_wall}{50}
		\cultnode{(1.5,0)}{colums_of_law}{50}
		\cultnodelbld{(3,0)}{orichalk_statue}{50}{2070}
		\cultnode{(0,-1.5)}{temple_of_posseidon}{50}
		\cultnode{(1.5,-1.5)}{yonaguni_pyramid}{50}
		\cultnode{(3,-1.5)}{atlantis_tablet}{50}
	\end{tikzpicture}
	Salvager get increased chance of finding Epic and Legendary artefacts.
	}		
	\tcbox[nodebox, %Northern Sagas
	raster multicolumn=1, top=5mm,
	raster multirow=3]{
		\begin{tikzpicture}
			\cultnode{(0,0)}{viking_edda}{50}
			\cultnode{(1.5,0)}{viking_nibelungen_ring}{50}
			\cultnodelbld{(3,0)}{viking_valkyrie_helmet}{50}{2205}
			\cultnode{(0,-1.5)}{viking_drakkar_carving}{40}
			\cultnode{(1.5,-1.5)}{viking_uppsala_altar}{50}
			\cultnode{(3,-1.5)}{ragnarsdrapa}{30}
		\end{tikzpicture}
		all warships:\\
     	+5\% maximum hitpoints\\ 
     	+3\% increased damage 
	}	
\end{tcbraster}


	\tcbline
\end{tcolorbox}
\pagebreak

\begin{tcolorbox}[height fill,
	raster height=\boxheight,
	page box]
	\include{garden.tex}
	\tcbline
\end{tcolorbox}
\pagebreak


\begin{landscape} % SIDEWAYS PAGE
	\begin{tcolorbox}[raster height=\boxheight,
		height=\paperwidth-2cm, % subtract margins manually
		page box]

		\section{Electrification \& Symbols}
		Almost all buildings in the \oldworld{}Old World and \newworld{} New World, excluding \textit{farms} and \textit{Lumberjacks}, can be electrified for up to \textit{200\%} productivity. \underline{Buildings marked with\smallicon{Electricity}require electricity by default}. \newworld{} New World \textit{animal farms} supply a bonus good when electrified, Farms building are marked with \smallicon{Electricity} and the base farm type.
		\\
		\\
		Some buildings have variants with different statistics, these are marked with \oldworld{}Old World, \newworld{} New World, \arctic{} Artic and \enbesa{} Enbesa.\\
		\\
		Goods that can be consumed by residences show these following values: Type of residence consuming this need, Consumption rate,  Workforce, \smallicon{credits}Income, \smallicon{happiness_positive}Happiness, \smallicon{research}Research points, \smallicon{heater}Heat.\\
		\\
		Consumption rate displayed is the amount of basic residences can be supplied by one 100\% factory. Hacienda Quarters and Skyscrapers have different rates. Skyscraper rates are also dependant on panorama effects. Items that give bonus residents also increase the consumption rate for those residences.\\
		\begin{tabular}{| l | l |}
		\hline
		Residence & consumption rate (relative to basic) \\
		\hline
		Hacienda Journalero Quarter & x2 \\
		Hacienda Obrero Quarter & x2 \\
		Hacienda Artista Quarter & x2 \\
		\hline
		Engineer level 1 & x[1.5 - 4] \\
		Engineer level 2 & x[2 - 4.5] \\
		Engineer level 3 & x[2.5 - 5] \\
		\hline
		Investor level 1 & x[1.5 - 4] \\
		Investor level 2 & x[2 - 4.5] \\
		Investor level 3 & x[2.5 - 5] \\
		Investor level 4 & x[3 - 5.5] \\
		Investor level 5 & x[3.5 - 6] \\
		\hline
		\end{tabular}\\
		
		
		\tcbline
		\section{Collaboration \& Thanks}
		If you would like to improve this document please check out the GitHub repository: \url{\thesourcerepository}\\
		\\
		Many thanks to all collaborators:
		\begin{itemize}[topsep=2pt, noitemsep]
			\renewcommand\labelitemi{--}
			\foreach \x/\y in {McDonnough/\url{https://github.com/McDonnough},René Kost/\url{https://github.com/rkost},kruzuahz/\url{https://github.com/kruzuahz}} {
				\item \x \space (\y)%
			}
		\end{itemize}
	\end{tcolorbox}
\end{landscape}

\end{document}